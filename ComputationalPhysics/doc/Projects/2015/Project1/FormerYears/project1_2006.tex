\documentclass[11pt,a4wide]{article}
\usepackage{verbatim}
\usepackage{listings}
\usepackage{graphicx}
\usepackage{epic}
\usepackage{eepic}
\usepackage{a4wide}
\usepackage{color}
\usepackage{amsmath}
\usepackage{amssymb}
\usepackage[dvips]{epsfig}
\usepackage{psfig}
\usepackage[T1]{fontenc}
\usepackage{cite} % [2,3,4] --> [2--4]
\usepackage{shadow}
\usepackage{hyperref}

\setcounter{tocdepth}{2}

\lstset{language=c++}
\lstset{alsolanguage=[90]Fortran}
\lstset{basicstyle=\small}
\lstset{backgroundcolor=\color{white}}
\lstset{frame=single}
\lstset{stringstyle=\ttfamily}
\lstset{keywordstyle=\color{red}\bfseries}
\lstset{commentstyle=\itshape\color{blue}}
\lstset{showspaces=false}
\lstset{showstringspaces=false}
\lstset{showtabs=false}
\lstset{breaklines}
\begin{document}
\section*{Introduction to numerical projects}

Here follows a brief recipe and recommendation on how to write a report for each
project.
\begin{itemize}
\item Give a short description of the nature of the problem and the eventual 
numerical methods you have used.
\item Describe the algorithm you have used and/or developed. Here you may find it convenient
to use pseudocoding. In many cases you can describe the algorithm
in the program itself.

\item Include the source code of your program. Comment your program properly.
\item If possible, try to find analytic solutions, or known limits
in order to test your program when developing the code.
\item Include your results either in figure form or in a table. Remember to
       label your results. All tables and figures should have relevant captions
       and labels on the axes.
\item Try to evaluate the reliabilty and numerical stability/precision
of your results. If possible, include a qualitative and/or quantitative
discussion of the numerical stability, eventual loss of precision etc. 

\item Try to give an interpretation of you results in your answers to 
the problems.
\item Critique: if possible include your comments and reflections about the 
exercise, whether you felt you learnt something, ideas for improvements and 
other thoughts you've made when solving the exercise.
We wish to keep this course at the interactive level and your comments can help
us improve it.
\item Try to establish a practice where you log your work at the 
computerlab. You may find such a logbook very handy at later stages
in your work, especially when you don't properly remember 
what a previous test version 
of your program did. Here you could also record 
the time spent on solving the exercise, various algorithms you may have tested
or other topics which you feel worthy of mentioning.
\end{itemize}



\section*{Format for electronic delivery of report and programs}
%
The preferred format for the report is a PDF file. You can also
use DOC or postscript formats. 
As programming language we prefer that you choose between C/C++ and Fortran90/95.
You could also use Java or Python as programming languages. 
Matlab/Maple/Mathematica/IDL are not allowed as programming
languages for the handins, but you can use them to check your results where possible.
The following prescription should be followed when preparing the report:
\begin{itemize}
\item Use Classfronter to hand in your projects, log in  at 
blyant.uio.no and choose 'fellesrom fys3150 og fys4150'.
Thereafter you will see an icon to the left with 'hand in' or 'innlevering'.
Click on that icon and go to the given project. 
There you can load up the files within the deadline.
\item Upload {\bf only} the report file and the source code file(s) you have developed.
The report file should include all of your discussions and a list of the codes you have developed. 
Do not include library files which are available at the course homepage, unless you have
made specific changes to them.
\item Comments  from us on your projects, approval or not, corrections to be made 
etc can be found under
your Classfronter domain and are only visible to you and the teachers of the course.
 
\end{itemize}
Finally, 
we do prefer that you work two and two together. Optimal working groups consist of 
2-3 students. You can then hand in a common report. 

\section*{Project 1, deadline 18 september 12pm (midnight)}

The aim of this project is to get familiar with various matrix operations,
from dynamic memory allocation to the usage of programs in the library
package of the course. 
For Fortran users memory handling and most matrix and vector operations
are included in the ANSI standard of Fortran 90/95. For C++ user however,
there are three possible options
\begin{enumerate}
\item Make your own functions for dynamic memory allocation of a 
vector and a matrix. Use then the 
library package lib.cpp with its header file 
lib.hpp for obtaining LU-decomposed matrices, solve linear equations
etc.
\item Use the library package lib.cpp with its header file 
lib.hpp which includes a function \verb?matrix? for dynamic memory
allocation. This program package includes all the other functions
discussed during the lectures for solving systems of linear equations,
obatining the determinant, getting the inverse etc.
\item Finally, we provide on the web-page of the course a library package
which uses Blitz++'s classes for array handling. You could then, since
Blitz++ is installed on all machines at the lab, use these classes for handling
arrays.
\end{enumerate}

Your program, whether it is written in C++ or Fortran 90/95, should include
dynamic memory handling of matrices and vectors. You should also read the matrix from a file
and write your results to a file. Make sure your code includes these options.
\begin{enumerate}
\item[(a)] 
Consider the linear system of equations 
%
\begin{eqnarray}
 a_{11}x_1 +a_{12}x_2 +a_{13}x_3 =&w_1 \nonumber \\
a_{21}x_1 + a_{22}x_2 + a_{23}x_3=&w_2 \nonumber \\
a_{31}x_1 + a_{32}x_2 + a_{33}x_3=&w_3. \nonumber 
\end{eqnarray}
This can be written in matrix form as
\[
   {\bf Ax}={\bf w}.
\]
Use the included programs for LU decomposition to solve the system of equations
\begin{eqnarray}
 -x_1 +x_2 -4x_3 =&0 \nonumber \\
  2x_1 + 2x_2 =&1 \nonumber \\
3x_1 + 3x_2 + 2x_3=&\frac{1}{2}. \nonumber 
\end{eqnarray}
Use first standard Gaussian elimination and compute the result
analytically. Compare thereafter your analytical results with
the numerical ones obtained using the LU programs in the program library.

\item[(b)] 
We start with the linear set of equations 
\[
   {\bf A}{\bf u} = {\bf f},
\]
where ${\bf A}$ is a tridiagonal matrix which we rewrite as 
\begin{equation}
    {\bf A} = \left(\begin{array}{cccccc}
                           b_1& c_1 & 0 &\dots   & \dots &\dots \\
                           a_2 & b_2 & c_2 &\dots &\dots &\dots \\
                           & a_3 & b_3 & c_3 & \dots & \dots \\
                           & \dots   & \dots &\dots   &\dots & \dots \\
                           &   &  &a_{n-2}  &b_{n-1}& c_{n-1} \\
                           &    &  &   &a_n & b_n \\
                      \end{array} \right)
\end{equation}
where $a,b,c$ are one-dimensional arrays of length $1:n$. 
We set the arrays $a$ and $c$ to be equal, namely $a_i=c_i=-1/100$.
We set $b_i = 2/100$.
Our linear equation reads
\begin{equation}
    {\bf A} = \left(\begin{array}{cccccc}
                           b_1& c_1 & 0 &\dots   & \dots &\dots \\
                           a_2 & b_2 & c_2 &\dots &\dots &\dots \\
                           & a_3 & b_3 & c_3 & \dots & \dots \\
                           & \dots   & \dots &\dots   &\dots & \dots \\
                           &   &  &a_{n-2}  &b_{n-1}& c_{n-1} \\
                           &    &  &   &a_n & b_n \\
                      \end{array} \right)\left(\begin{array}{c}
                           u_1\\
                           u_2\\
                           \dots \\
                          \dots  \\
                          \dots \\
                           u_n\\
                      \end{array} \right)
  =\left(\begin{array}{c}
                           f_1\\
                           f_2\\
                           \dots \\
                           \dots \\
                          \dots \\
                           f_n\\
                      \end{array} \right).
\end{equation}
A tridiagonal matrix is a special form of banded matrix where all the elements are zero except for 
those on and immediately above and below the leading diagonal.
The above tridiagonal system   can be written as
\begin{equation}
  a_iu_{i-1}+b_iu_i+c_iu_{i+1} = f_i,
\end{equation}
for $i=1,2,\dots,n$. We see that $u_{-1}$ and $u_{n+1}$ are not required and we can set $a_1=c_n=0$.
The algorithm for solving this set of equations is rather simple and requires two steps only, a decomposition 
and forward substitution and finally a backward substitution. 
A possible part of the code which solves these equations is, with the forward substitution part first
\lstset{language=c++} 
\begin{lstlisting}
   btemp = b[1];
   u[1] = f[1]/btemp;
   for(i=2 ; i <= n ; i++) {  
      temp[i] = c[i-1]/btemp;
      btemp = b[i]-a[i]*temp[i];
      u[i] = (f[i] - a[i]*u[i-1])/btemp; 
   }  
\end{lstlisting}
Note that you should avoid cases with $b_1=0$. If that is the case, you should rewrite the equations
as a set of order $n-1$ with $u_2$ eliminated. 
Finally we perform the backsubstitution leading to the following code
\begin{lstlisting}
   for(i=n-1 ; i >= 1 ; i--) {  
      u[i] = u[i]-temp[i+1]*u[i+1];
   }  
\end{lstlisting}

Your first task is to find the number of operations needed to solve the above equations. Show that they behave like $O(n)$ with 
$n$ the dimensionality of the problem. Compare this with standard Gaussian elimination and LU decomposition.  
Then you should code the above algorithm and solve the problem for a $100\times 100$ matrix where you let the vector $f_i$ take
values $f_1=0.1,f_2=0.2,\dots f_{100}=10$. Compare your results with those from the LU decomposition codes.
Use for example the unix function {\em time} when you run your codes and compare the time usage between LU decomposition and
the tridiagonal solver. Comment your results.
\end{enumerate}


\end{document}





















