%\documentclass[]{revtex4}

%\documentclass[preprint,showpacs,preprintnumbers,amsmath,amssymb]{revtex4}

% Some other (several out of many) possibilities
\documentclass[preprint,aps]{revtex4}
%\documentclass[preprint,aps,draft]{revtex4}
%\documentclass[prb]{revtex4}% Physical Review B

%\usepackage{amsmath}
%\usepackage{graphicx}% Include figure files
%\usepackage{dcolumn}% Align table columns on decimal point
%\usepackage{bm}% bold math

\begin{document}

\title{Realistic three-nucleon effective interactions in nuclear structure studies}

\author{M.~P.~Kartamyshev, M.~Hjorth-Jensen, T.~Engeland and E.~Osnes}

\affiliation{Department of Physics and Center of Mathematics for Applications, 
University of Oslo, N-0316 Oslo, Norway}

%\date{\today}

\begin{abstract}
The role of effective three-nucleon interaction in shell-model and nuclear structure calculations
is still an unsettled question. In constrast to the case of the light nuclei,
where the role of the three-nucleon forces seems to be rather well established, none of the present
calculations of effective interactions for medium and heavy nuclei are sufficiently complete to
enable one to draw definite conclusions about the role of the effective three-nucleon interactions. 

Starting from the folded-diagram theory of Kuo and collaborators, we construct an effective
three-nucleon interaction originating from the two-nucleon force. Influence of the three-nucleon
terms on nuclear properties is investigated in shell-model studies of selected nuclei in $^{16}$O,
$^{40}$Ca and $^{100}$Sn mass regions.
\end{abstract}

%\pacs{21.60.-n; 21.60.Cs; 24.10.Cn; 27.60.+j }

\maketitle

\end{document}