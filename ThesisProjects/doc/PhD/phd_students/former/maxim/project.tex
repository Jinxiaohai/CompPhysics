
\documentclass{article}


\begin{document}

\title{Dr.Scient Thesis project: Three-body forces in nuclei}

\author{Maxim Kartamyshev\\ 
        Department of Physics, University of Oslo} 

\maketitle  



\section{Introduction}

The main aim of 
this thesis project is to study the inclusion of real three-body forces
in the nuclear many-body problem with emphasis on shell-model studies
of finite nuclei.

The inclusion of 
these three-body forces in a many-body context will be achieved through
the derivation of an effective three-body interaction to be used
in large-scale shell-model calculations of properties of finite 
nuclei. Two main many-body approaches to the derivation of the effective interaction will
be used and studied in this thesis:
\begin{enumerate}
\item Coupled Cluster method with various approximations
\item Green's function methods
\end{enumerate}

The Oslo group has already developed a parallel shell-model code
which allows for the use of three-body effective interactions.
The task of this thesis is thus to include recent models for three-body 
forces and compute effective interactions based on the models for the 
nucleon-nucleon interaction and three-body forces. 

There are several features of nuclei and infinite nuclear matter which
point to the need of real three-body forces.
In infinite nuclear matter they are needed in order to reproduce properly
the binding energy and saturation density of symmetric nuclear matter,
as pointed out in Refs.~\cite{vijay98,hh2000}. 

For finite nuclei one  needs a real three-body force in order to reproduce
the spin-orbit splitting between the single-particle orbits in e.g., $^{15}$N
\cite{vijay92}. There is indeed a wealth of indications from experiment and theory   
that point to the fact that a real three-body may be needed in shell-model studies. The ground state of $^{10}$B has a spin assignement of $J=3^+$.
Calculations based on Green's function Monte Carlo methods or no-core
shell-model calculations with two-body forces yield a ground state of
$J=1^+$. The inclusion of a real three-body force gives the correct
spin assignement \cite{erich2002}. 
Shell-model studies of the chain of tin isotopes reveal also that
the binding energy trend of these nuclei is not correctly 
reproduced, see e.g., Ref. \cite{anne98}.

Thus, the study of a real three-body in nuclear structure calculations
is both timely and needed. 
In the next section we briefly escribe the theoretical approaches to be employed.
\section{Project description}
The potential models to be used, both two-body and three-body, are based
on a meson-exchange picture. These potential models are thereafter 
parameterized
in order to reproduce the two-body scattering data and selected features
of light nuclei and nuclear matter.
We will use the potential models of the Argonne group, as detailed in 
Ref. \cite{threeb}.
The thesis project consists thence of the following steps
\begin{enumerate}
\item In order to get used to the two and three-body interaction models of
the Argonne group, the potentials are used in a Faddeev calculation in order
to reproduce previous calculations of the binding energy of the triton.
\item The next step is 
the transformation to a  
harmonic oscillator basis for the three-body force in an angular momentum
coupling scheme.
This step serves as an exercise
for constructing  a three-body wave function which obeys the Pauli principle
using an angular momentum coupling scheme.
\item The next step is the construction of a two and three-body $G$-matrix
in a truncated Hilbert space. This space is however large enough so that one
can sum up large classes of diagrams in the coupled-cluster 
and Green's function methods, as detailed in the next  steps.
The $G$-matrix serves to renormalize the hard core of the two and three-body
interactions.
We have recently \cite{khj2002} finished work on a parallel two-body
$G$-matrix for a large space, with a cutoff in a harmonic oscillator basis
of $2n+l > 30$. This program will then be extended in order to also
include a three-body force and thereby obtain a three-body $G$-matrix.

\item Recently, Dean and Hjorth-Jensen \cite{dhj2002} have performed 
Coupled cluster calculations with a two-body $G$-matrix for a large 
model space, $2n+l > 10$. With center of mass corrections included,
these two-body calculations agree well with results from other many-body
schemes with two-body forces only for $^4$He.
This thesis deals with the inclusion of more complicated 
correlations in the coupled
cluster scheme, with and without a three-body $G$-matrix.

The plan is to compute the binding energy for several 
closed-shell nuclei and to
derive effective interactions for valences space systems. The latter will
be included in large-scale shell model calculations in order to test
the role played by three-body forces in nuclear spectroscopy.
This step will be achieved using the Oslo shell-model code, a parallel code
which allows for the use of effective three-body interactions as well.
All codes are written in the proton-neutron formalism and allow for studies
of nuclei with differing proton and neutron numbers. The Coulomb force
will also be included in these calculations.

\item The Oslo group has also a long experience on constructing
effective interactions for the shell model based on perturbative
many-body approaches (MBPT). A critical comparison of the Coupled cluster
method with such a many-body scheme forms also a part of this thesis
project.

\item The last topic is the computation of the binding energy of closed
shell nuclei and the derivation of valence space effective interactions
from Green's function methods. In particular we will focus on the derivation
of so-called Parquet diagrams, see e.g., Ref.\ \cite{mhj1998}. 
At present, using the same large-space two-body $G$-matrix as in Coupled
cluster calculations, we are able to sum diagrams with intermediate particle-particle, hole-hole and particle-hole states. 
These calculations will be compared with the Coupled-cluster calculations 
and the MBPT calculations. This has never been performed before at the two-body level and may yield new insights in many-body theory.
The ultimate aim is to include a three-body $G$-matrix in these calculations as
well.
\end{enumerate}

\section{Working conditions}
The work will be done at  the theoretical 
nuclear physics group of the
Department of Physics, University of Oslo. This
group has a long-standing experience in many-body physics
and has worked intensively on shell-model topics, effective 
interactions for the nuclear shell model, 
physics of dense matter and computational quantum mechanics in general. 
The group has also a large experience on high-performance computing topics
and has recently developed parallel shell-model codes which 
also include three-body interactions. Parallel codes for effective two-body
diagrams have recently been developed \cite{khj2002}.

At presently the group counts  two professors (Hjorth-Jensen and Osnes), 
one retired but still active professor (Engeland), 
three Phd students 
and 6 Msc students
working on various
aspects of many-body physics and nuclear physics. A post-doc
(Garbarino) starts working with the group in january 2004.
The research group participates in the newly established
center of excellence at the University of Oslo, Mathematics
for applications, see www.cma.uio.no for more information.
The group has extensive collaborations with the subatomic physics
group at the University of Bergen.


In addition, the theory activity in Oslo has extensive
international collaborations with several Universities and research 
laboratories in the USA, Europe and Japan. 


\begin{thebibliography}{}
\addcontentsline{toc}{section}{References}


\bibitem{vijay98} A.\ Akmal, V.R.\ Pandharipande and D.G.\ Ravenhall,
Physical  Review  C 58 (1998) 1804.
\bibitem{hh2000}
    H.\ Heiselberg and M.\ Hjorth-Jensen,
    {\em Phases of dense matter in neutron stars},
    Physics Reports {\bf 328} (2000) 237-327 
\bibitem{vijay92}Steven C. Pieper, V. R. Pandharipande, 
{\em Origins of spin-orbit splitting in}$^{15}$N, Physical Review Letters 70, 2541-2544 (1993)

\bibitem{erich2002} E. Ormand and P. Navratil, 
{\em Ab Initio Shell Model Calculations with Three-Body Effective Interactions for p-Shell Nuclei}, Physical Review Letters 88, 152502 (2002)
\bibitem{anne98} A.\ Holt, T. Engeland, M. Hjorth-Jensen, and E. Osnes,
    {\em Shell-model calculations of heavy} Sn {\em isotopes}, 
    Nuclear Physics {\bf A634} (1998) 41-56
\bibitem{threeb}  Steven C. Pieper, V. R. Pandharipande, 
                  R. B. Wiringa, and J. Carlson, {\em Realistic models of pion-exchange three-nucleon interactions}, Physical Review C 64, 014001 (2001)
\bibitem{khj2002} M.~P.~Kartamyshev and M. Hjorth-Jensen, {\em A parallel code
for two-body effective interactions based on Green's function methods}, 
in preparation for Comp.~Phys.~Com.
\bibitem{dhj2002} D.~J.~Dean and M.~Hjorth-Jensen, {\em Coupled cluster
approach to nuclear physics}, preprint nucl-th/0308088 and submitted to
Phys. Rev. C; K.~Kowalski, D.~J.~Dean, M.~Hjorth-Jensen, T.~Papenbrock and 
P.~Piecuch, {\em Coupled cluster
calculations of ground and excited states of nuclei}, in preparation for
Phys.~Rev.~Lett.
\bibitem{mhj1998} D.~J.~Dean, T.~Engeland, M.\ Hjorth-Jensen, 
M.~Kartamyshev and E.~Osnes, {\em Effective interactions
and the nuclear shell model}, Review article by invitation of the editor,
Prof.~Amand Faessler and to appear in 
Prog.~Part.~Nucl.~Phys. (2004).

\end{thebibliography}

\end{document}


