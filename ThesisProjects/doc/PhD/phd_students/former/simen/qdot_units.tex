\documentclass{article}
\usepackage{a4wide}
\usepackage[latin1]{inputenc}
\usepackage{amsmath,latexsym}

\begin{document}

\section*{Dimensionless units for the h.o. basis}

We wish to study the transition from regular units to dimensionless
units for a two-particle harmonic oscillator system with Coulomb
interaction. We also include a magnetic field in the discussion. Consider the
Hamiltonian
\[ H = \sum_{i=1}^2 \left[\frac{1}{2m}\left(-i\hbar\nabla - e\vec{A}\right)^2 +
\frac{1}{2}m\omega_0^2 r_i^2 + V(\vec{r}_i) \right] + C, \] where $V$
is a perturbation of the harmonic potential, and where $C$ is the Coulomb
interaction, viz.,
\[ C = \frac{e^2}{4\pi\epsilon_0\epsilon}\frac{1}{r_{12}}, \]
in which
\[ \frac{e^2}{4\pi\epsilon_0} \approx 1.440 \text{ eVnm}. \]
The parameters to the model are thus
\begin{itemize}
\item Magnetic field strength $\gamma$. We intend to supply this in
  units of Tesla, i.e., $\gamma = \hat{\gamma}\:\mathrm{T} = \hat{\gamma}\:\mathrm{kg\cdot
  s^{-1} \cdot C^{-1}}$.
The magnetic field is given as
\[ \vec{B} = \nabla\times \vec{A} = \gamma\vec{e}_z. \]
We use the Coulomb gauge, in which the (otherwise nonunique) potential
$\vec{A}$ is given by
\[ \vec{A} = \frac{\gamma}{2}(-y,x,0). \]
\item (Effective) mass $m$, in units of electron masses:
\[ m = \tilde{m}m_e, \quad m_ec^2\approx 511000 \:\mathrm{eV}. \]
\item Harmonic potential confinement length $a$. This is the ``trap
  size'', i.e., the characteristic length of the harmonic oscillator
  potential, viz,
\[ a = \sqrt{\frac{\hbar}{m\omega_0}} \Leftrightarrow \omega_0 =
  \frac{\hbar}{ma^2}. \]
\item Dielectric constant $\epsilon$; a dimensionless parameter.
\end{itemize}

We know that introducing the length scale $a$ and the energy unit
$\hbar\omega_0$ in the harmonic oscillator with $\gamma=0$ will render it
dimensionless and on a particularly simple form. However, the magnetic
field will alter the confinement length and hence the energy
scale. In the following, we ignore the perturbation $V$. Its inclusion is
easy once we have established the units.

\newcommand{\LZ}{\left(-\mathrm{i}\frac{\partial}{\partial \phi_i}\right)}
Writing out the kinetic term in the Hamiltonian, we obtain
\[ H = \sum_{i=1}^2 \left[-\frac{\hbar^2}{2m}\nabla_i^2 +
  \frac{e^2\gamma^2}{8m}r_i^2 - 2\frac{\hbar e\gamma}{4m}\LZ +
  \frac{1}{2}m\omega_0^2r_i^2 \right] + C. \]
We observe that if we define
\[ \Omega \equiv \sqrt{\omega_0^2 + \frac{e^2\gamma^2}{4m^2}}, \]
we get
\[ H = \sum_{i=1}^2 \left[-\frac{\hbar^2}{2m}\nabla_i^2 +
  \frac{1}{2}m\Omega^2 r_i^2 - \frac{\hbar e\gamma}{2m}\LZ 
   \right] + C. \]
We now introduce the length and energy units:
\[ H = \hbar\Omega\tilde{H}, \quad \vec{r}_i =
   \sqrt{\frac{\hbar}{m\Omega}}\vec{x}_i. \]
The dimensionless Hamiltonian becomes
\[ \tilde{H} = \sum_{i=1}^2 \left[-\frac{1}{2}\tilde{\nabla}_i^2 +
  \frac{1}{2} x_i^2 - \frac{e\gamma}{2m\Omega}\LZ 
   \right] + \tilde{C},\]
where
\[ \tilde{C} = \frac{1}{\hbar\Omega}C  =
   \frac{e^2}{4\pi\epsilon_0\epsilon}\sqrt{\frac{m}{\hbar^3\Omega}}\frac{1}{x_{12}}
   \]
is dimensionless. Note that a uniform scaling of $\vec{r}$ will not
alter the angular momentum operator $\LZ$.

Let ut introduce a proper scale for the magnetic field strength
$\gamma$. A natural combination is
\[ \gamma = \tilde{\gamma}\frac{\hbar\omega_0}{\mu_B} =
\tilde{\gamma}\frac{2m_e\omega_0}{e}, \]
where 
\[ \mu_B = \frac{e\hbar}{2m_e} \approx
5.7884\cdot10^{-5}\:\mathrm{eV\:T^{-1}} \]
is the Bohr magneton. Hence, $\tilde{\gamma}$ is dimensionless. We then obtain
\[ \gamma = \frac{\tilde{\gamma}}{\tilde{m}} \frac{2\hbar}{ea^2}
\approx \frac{\tilde{\gamma}}{\tilde{m}} 1316.4 \: \mathrm{T}
\frac{\mathrm{nm}^2}{a^2}. \]
In other words, the chosen unit for magnetic field strength depends on
the confinement strength and the mass. This is no surprise: The
Lorentz force on a moving charge depends on the mass, and we
introduced the length scale on this motion in the choice of units.

This also yields a particularly simple expression for $\Omega$, viz,
\[ \Omega = \omega_0 \sqrt{1 +
  \frac{\tilde{\gamma}^2}{\tilde{m}^2}}. \]
Moreover, we obtain
\[ \tilde{C} = \frac{e^2}{4\pi\epsilon_0}\frac{\tilde{m}a}{\epsilon}
\frac{m_ec^2}{(\hbar c)^2}
\frac{1}{\sqrt[4]{1+\tilde{\gamma}^2/\tilde{m}^2}}\frac{1}{x_{12}}. \]

For completeness, assume now that $\gamma$ is given in units of Teslas, i.e., $\gamma =
\hat{\gamma}\:\mathrm{T}$. We then obtain
\[ \tilde{\gamma} = \frac{\hat{\gamma}}{1316.4} \tilde{m}
\frac{a^2}{\textrm{nm}^2}. \]

Inserting $\tilde{\gamma}$ into the Hamiltonian we obtain
\[ \tilde{H} = \sum_{i=1}^2 \left[-\frac{1}{2}\tilde{\nabla}_i^2 +
  \frac{1}{2} x_i^2 - \alpha\LZ \right] + \lambda\frac{1}{x_{12}}, \]
with
\[ \alpha \equiv \frac{\tilde{\gamma}}{\tilde{m}} \frac{1}{\sqrt{1 +
    \tilde{\gamma}^2/\tilde{m}^2}} \]
and
\[ \lambda \equiv \frac{e^2}{4\pi\epsilon_0}\frac{\tilde{m}a}{\epsilon}
\frac{m_ec^2}{(\hbar c)^2}
\frac{1}{\sqrt[4]{1+\tilde{\gamma}^2/\tilde{m}^2}} \approx 18.903
\frac{\tilde{m}a\:\mathrm{nm}^{-1}}{\epsilon} \frac{1}{\sqrt[4]{1+\tilde{\gamma}^2/\tilde{m}^2}}.\]


\subsection*{Remarks}

Consider for the moment the case $\gamma=0$, so that $\Omega=\omega_0$.
Notice that $C$ and $\tilde{C}=C/\hbar\omega_0$ scale differently with increasing
confinement length $a$. Intuitively, with large $a$, the particles will
spend more time away from each other, weakening the interaction on
average. This agrees with $A \sim a^{-1}$. However, notice that the
energy level spacing $\hbar\omega \sim a^{-2}$ gets smaller with
larger $a$, i.e., the spectrum approaches a continuous one. This
implies that the \emph{scaled} interaction 
$\tilde{A}^ \sim a$ becomes \emph{stronger} with increasing $a$.

There is no way around this problem. Seemingly, we could use a basis
of Hermite functions falling off faster, making the Coulomb matrix
elements smaller. However, this will destroy the asymptotics of the
wave function (it falls off too fast). Moreover, the h.o. part is no
longer diagonal, implying the need of a bigger basis to obtain
convergence. 



\end{document}
