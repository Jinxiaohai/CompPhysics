
Temaet for Simen 
 sitt drgrads prosjekt dreier seg om
numeriske metoder for kvantemekaniske systemer, 
med spesiell vekt p� den tidsavhengige Schroedingerlikningen 
for systemer med f� partikler eller midlere felt likninger som den ikke-lin�re
Gross-Pitaeviiski likningen, h�yst aktuell i studier av Bose-Einstein
kondensat.
Under hovedfaget studerte Simen
numeriske l�sninger av tidsavhengige systemer, 
som f.eks atomer i ulike dimensjoner og geometrier 
under p�virkning av ultra-korte laserpulser, 
geometriske invarianter for Schr\"dingerlikningen, som f.eks gauge-invarians, 
og l�sning av store egenverdisystemer som oppst�r i kvantemekanikk.
Dette er temaer som �nskes videref�rt i drgrads arbeidet og er sv�rt aktuelle 
i atomfysikk og faste stoffers fysikk samt numerisk matematikk.
Det matematisk numerisk verkt�yet Simen skal ta i bruk er basert p�
'Finite Element' metoden. Prof. Ragnar Winther ved CMA 
vil v�re medveileder p� oppgaven. Han er en internasjonalt annerkjent ekspert i numerisk
matematikk og 'Finite Element' metoden i s�rdeleshet.

Jeg har kjent Simen siden h�sten 2000, da som laveregrads student, 
og i kraft av � v�re veiledere p� hovedfaget fra h�sten 2002 til mars 2004.
Ragnar Winther har kjent Simen i godt og vel ett �r.
Uten � overdrive, tror jeg begge to kan underskrive p� at
Simen er en eksepsjonell student. Han blei ansatt som stipendiat p� CMA umiddelbart
rett etter hovedfagseksamen, i skarp konkurranse med sv�rt gode 
nasjonale og internasjonale
kandidater.
Hovedfaget hans var mildt sagt imponerende og han utviste en stor vitenskapelig 
kreativitet og modenhet som er sjelden til og med blant drgrads studenter.
Simen mestrer analytiske metoder i fysikk og matematikk, numerisk matematikk 
samt at han er meget oppeg�ende i moderne programmeringsverkt�y.
Dette kombinert med masse initiativ, ideer og kreativitet gj�r at Simen 
vil passe ypperlig inn i forskningsprofilen
til CMA og aktiviteten i Computational Physics ved Fysisk Institutt.
I h�st har Simen, sammen med Per Christian Moan som er postdoc p� CM, starta en 
forelesningsserie i Computational Quantum Mechanics, se 
http://www.cma.uio.no/seminars/2004quant_mech.html for flere detaljer. 
Sammen med forskere, PhD og Master studenter p� Fysisk Institutt og CMA utgj�r dette 
allerede et viktig forskningselement ved CMA. 
Sammen med Per Christian Moan holder Simen p� � forfatte et st�rre arbeid om
stabiliteten til 'Finite Element' metoden anvendt p� Hamiltonske systemer.

I sum, Simen er en ypperlig kandidat med s�rdeles gode faglige kvalifikasjoner.

