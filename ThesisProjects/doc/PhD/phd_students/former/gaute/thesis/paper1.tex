\chapter{Paper1}
\section{Introduction to Paper I} 
Paper I discusses how the momentum space Schr\"odinger equation
may be analytically continued to the second energy sheet. 
The method of anlytical continuation is based on 
deforming (distoting) the integration contour, and this method
is commonly known as the Contour Deformation Method (CDM). 
The rules for analytical continuation of integral equations
are discussed, and as an example the Schr\"odinger equation for the
Malfliet-Tjon potential, which is a nucleon-nucleon 
potential consisting of Yukawa terms, is analytically continued 
to the second Riemann sheet. 
It is found, by choosing a suitable deformed contour (rotation+translation), 
that the analytical  structure of the Malfliet-Tjon 
potential allows for a continuation into the third
quadrant of the complex $k$-plane, and consequently 
the Contour Deformation Method allows for a study 
of virtual states as well as decaying resonances. Further, CDM 
is an alternative approach to the full solution of the off-shell
scattering amplitude. In the case of potentials
consisting of Yukawa terms, choosing a rotated+translated
contour which avoid the singularities of the potential, 
allows for a complete solution of the $t$-matrix 
in a large momentum range. Not only does CDM give us information 
of the complete pole structure of the scattering matrix 
but also the scattering amplitude is obtainable by expanding
the Green's function in a complete set of Berggren states.
Expanding the Green's function in a complete set of states, 
consisting of bound, resonant and non-resonant continuum states
allows for a separate study of the resonant contributions 
to the scattering amplitude. Disentangling the resonant
behaviour of the scattering amplitude from the 
smooth continuum backrground is gives interesting
insights, since the most interesting process taking
place in the continuum is the production of resonance  
phenomena. Finally CDM is applied to the 
CD-Bonn  interaction, which is a realisitic nucleon-nucleon
potential, and virtual states in the isospin triplet channel 
$^1S_0 $ are solved for.  
  
\newpage
\section{\it The contour deformation method in momentum space, applied to subatomic physics} 
\label{sec:paper1}
\vskip 2.cm
{\Large G.~Hagen, J.~S.~Vaagen and M.~Hjorth-Jensen \\[0.5cm]
  J. Phys. A: Math. Gen., {\bf 37}, 8991 (2004).}


