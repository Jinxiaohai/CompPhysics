\chapter{Introduction}
\label{introduction}

The formulation and development of quantum theory in the first half of
the $20^{\mathrm{th}}$ century has led to a revolution in our
understanding of fundamental physics. Quantum theory has demonstrated
surprising accuracy and predictive power, and the importance of
quantum theory is unchallenged. The Schr\"odinger equation, which is
the fundamental equation of quantum mechanics, cannot be solved
analytically for any but the most trivial of systems. Numerical
many-body approaches provide powerful tools for solving this
equation.
\newline
%
\newline
In electronic structure calculations the treatment of
electron-electron interactions is the main source of
difficulty. These interactions cannot easily be separated out or
treated without approximation. Quantum Monte Carlo (QMC) methods treat
electron-electron interactions almost without approximation. The
accuracy of the QMC methods enables a great deal of confidence to be
placed in the results obtained.
\newline
%
\newline
In this thesis the basics of a Variation Monte Carlo (VMC) algorithm
are implemented and applied to several atoms. The VMC method forms the
basis of the QMC machinery. VMC used together with Diffusion Monte
Carlo (DMC) provides a powerful tool for incorporating correlation
effects into the many-body wave-function, and by means of Monte Carlo
(MC) integration the expectation values of different physical
observables can be obtained.
\newline
%
\newline
Before formulating the VMC algorithm several concepts
must be understood. In chapter \ref{AtomicPhysics} the atomic problem
is formulated, and then approximated to a form suited for numerical
calculations. The solutions to the one-electron hydrogen atom are
presented, before we introduce the variational and perturbative
methods to the helium atom.
\newline
%
\newline
In chapter \ref{NumericalApproaches}, different numerical approaches
are studied in greater detail. Both Hartree-Fock (HF) and Density
Functional Theory (DFT) are commonly used methods. Hartree-Fock will be
studied thoroughly, both because HF-solutions are used as a basis for
our VMC calculations and because understanding the HF method provides
useful insights to the other many-body approaches. DFT is particularly
interesting for larger system (100 particles or more), and will only
be mentioned briefly. 
\newline
%
\newline
The Configuration Interaction (CI) method is a natural extension of the
HF method, and the principles behind it are easy to understand.
The principal shortcoming of the CI
method is that it does not provide a compact description of the electron
correlation, and has a large growth in the number of configurations
needed for large systems, see ref. \cite{helgaker2002}. Among the
different many-body methods, Coupled-Cluster (CC) represents the most
effective method for atomic systems. Although computationally very
fast for small systems, it is practically impossible to carry out for
large systems due to the power six (or higher) scaling with the number
of particles involved. M\o ller-Plesset Many-Body Perturbation Theory
(MPPT) provides a different route to the solution of the Schr\"odinger
equation allowing us to approach the exact solution in a systematic
fashion based on an order-by-order expansion of the wave-function and
the energy. For systems where several important configurations are
present, the Multiconfiguration Self-Consistent Field method (MCSCF)
is well suited, but a problem with this method is that it shows no
clear sign of convergence.
\newline
%
\newline
A completely different approach to the many-body problem is realized
through Monte Carlo integration. Monte Carlo methods make
the evaluation of high dimensional integration possible. In the
Variational Monte Carlo (VMC) approach \emph{any} trial wave-function
may be optimized with respect to either energy or variance
minimization. This method therefore allows flexibility beyond the
orbital representation. Furthermore, accurate results are obtained for
several systems. A particularly interesting extension to
the VMC method is to combine VMC with the Diffusion Monte Carlo (DMC)
method. DMC yields in principle an exact solution to the many-body
problem. In practice, however, the DMC method requires a guiding
function, that accurately represents the basic features of the
eigenfunction, as input. Commonly, 
a variational trial wave-function is optimized by VMC, and the
optimized function provides the guiding function for DMC
calculations. The accuracy of DMC is limited only by this guiding
function. Therefore, development of the VMC method should be seen in
this context, and together with DMC it provides a highly efficient
tool for solving quantum mechanical problems with many
particles. Furthermore, the VMC and DMC methods share 
several similarities, and in developing the VMC method the foundation of
the DMC method is established. Nevertheless, VMC is also an efficient
tool on its own, and for smaller systems much of the 
electronic correlations, approximated in for example a HF
calculations, are regained.
\newline
%
\newline
In chapter \ref{Implementation} the program code developed through this
thesis is outlined. The program is coded in C++, and consists of
about 7 000 lines. A detailed description of the code has therefore
been omitted. Instead we focus on the theoretical principles regarding
the numerical optimization of the key building blocks of the
program. The main effort of this thesis has been developing the
code, and a few examples are provided to help clarify some of the
underlying technicalities regarding its implementation.
\newline
%
\newline
In chapter \ref{Results} results produced by the code are
presented. Tests regarding consistency with known solutions are
presented, as well as stability tests regarding wave-function
optimization schemes. Such tests are essential in the development of
any numerical tool, and have therefore been given special
attention. Step-by-step improvement of trial wave-functions are
also outlined, and results including electron-electron correlations
are presented for several atoms; from helium with its two electrons to
argon with a total of eighteen electrons. The results indicate the
efficiency of the numerical implementation presented here. The results
also indicate that electron-electron correlations become
less important as the number of electrons
increases. Therefore, inclusion of additional correlation effects is
needed in making accurate calculations also for larger systems. A
further development is the sophistication of the trial wave-functions.
\newline
%
\newline
To be able to effectively optimize more complicated trial
wave-functions the efficiency of the VMC algorithm needs further
development. In particular, we will study auto-correlation effects
which are one of the key limiting factors of the QMC methods. After
investigation of these effects we suggest ways to decrease them.
%We also look at possible extensions to larger systems such as
%molecules.
\newline
%
\newline
Finally, the insights gained through the development of this
thesis will be presented, together with suggestions of further
development of the program code.
\newline
%
\newline
Note that in this thesis it is assumed that the reader is familiar
with the basics of quantum mechanics. For introductory quantum
mechanics consult for example
refs. \cite{shankar1994,hemmer1980,atkins2003,bransden1983}. 
