\chapter{Conclusion}

In this thesis we have worked out the basic principles regarding the
use of Monte Carlo integration to solve many-body quantum mechanical
systems. A fast and reliable code has been developed, and principal
advances for future development have been identified. In this section
we summarize the insights gained through this work.
\newline
%
\newline
Quantum Monte Carlo methods provide powerful tools for solving the
many-body Schr\"o\-dinger equation. As a part of developing the QMC
machinery, Variation Monte Carlo (VMC) is a natural starting point. 
Here the basics of the VMC method have been developed, and the method 
has been applied to the atomic problem. The results given in the
previous chapter indicate that the main building blocks of the VMC
method are very efficient. Furthermore, the principal limitations
have been identified, and suggestions for future work have been
outlined. 
\newline
%
\newline
Several key factors regarding the program code need further
development. The main factor is related to auto-correlation
effects. The Metropolis 
algorithm needs to be sophisticated so that the walkers span the
phase space more efficiently. We further identified the
need for different step lengths at different length scales; the
electrons close to the nucleus should be moved shorter than the
electrons further away. A suggestion was made regarding possible
labeling of the individual electrons, as the different electron
permutations of the probability distribution simply resulted in
symmetrical subspaces.
\newline
%
\newline
Another key factor for future development is to allow greater
flexibility in the variational form of the trial wave-function; to
allow electron-nucleus and electron-electron-nucleus correlations and
also to allow linear combinations of Slater determinants.
\newline
%
\newline
An additional and natural extension is to implement the Diffusion
Monte Carlo (DMC) method. This method produces excellent results for
the electronic energies, and its foundations are laid when developing
the VMC method. Here an ensemble of walkers is moved, and making many
walkers is a simple matter. The diffusion, drift and branching terms
should also be simple to implement.
\newline
%
\newline
The energy and variance optimization schemes have been
studied, but we have omitted to mention effective procedures for
optimization of \emph{many} parameters simultaneously. High-dimensional
optimization is a common procedure in many areas, and can be
incorporated into the variance optimization scheme. The basic building
blocks of the variance optimization scheme has been developed and can
be used even for many-dimensional parameter optimization. However,
such procedures requires high accuracy in the different variations.
This emphasizes the importance of reducing auto-correlation effects
and obtaining better trial wave-functions.  Another easy and natural
extension to the current program is to parallelize the code also for
variance optimization.
\newline
%
\newline
The above extensions would result in a very efficient tool for
many-body quantum mechanics, that can be applied to a wide variety of
problems. For example for materials and molecules the Slater
determinant can be reduced to a block diagonal matrix by identifying
non-overlapping parts. Similarly, the terms in the Jastrow
factor approach a constant with increasing distance, and such terms
may also be removed. This procedure results in a linear dependency
with the number of atoms involved. For large molecules an interesting
application could be to apply the VMC method to Density Functional
Theory (DFT) determinants.
\newline
%
\newline
The VMC method is one of the several quantum mechanical many-body
methods. In particularly, the VMC method provides an easy and
efficient tool for including particle-particle correlations based on
physical principles. As one of the main building blocks of the QMC
machinery it deserves further investigation!

%\newline
%\begin{figure}[hbtp]
%\begin{center}
%  % GNUPLOT: plain TeX with Postscript
\begingroup
  \catcode`\@=11\relax
  \def\GNUPLOTspecial{%
    \def\do##1{\catcode`##1=12\relax}\dospecials
    \catcode`\{=1\catcode`\}=2\catcode\%=14\relax\special}%
%
\expandafter\ifx\csname GNUPLOTpicture\endcsname\relax
  \csname newdimen\endcsname\GNUPLOTunit
  \gdef\GNUPLOTpicture(#1,#2){\vbox to#2\GNUPLOTunit\bgroup
    \def\put(##1,##2)##3{\unskip\raise##2\GNUPLOTunit
      \hbox to0pt{\kern##1\GNUPLOTunit ##3\hss}\ignorespaces}%
    \def\ljust##1{\vbox to0pt{\vss\hbox to0pt{##1\hss}\vss}}%
    \def\cjust##1{\vbox to0pt{\vss\hbox to0pt{\hss ##1\hss}\vss}}%
    \def\rjust##1{\vbox to0pt{\vss\hbox to0pt{\hss ##1}\vss}}%
    \def\stack##1{\let\\=\cr\tabskip=0pt\halign{\hfil ####\hfil\cr ##1\crcr}}%
    \def\lstack##1{\hbox to0pt{\vbox to0pt{\vss\stack{##1}}\hss}}%
    \def\cstack##1{\hbox to0pt{\hss\vbox to0pt{\vss\stack{##1}}\hss}}%
    \def\rstack##1{\hbox to0pt{\vbox to0pt{\stack{##1}\vss}\hss}}%
    \vss\hbox to#1\GNUPLOTunit\bgroup\ignorespaces}%
  \gdef\endGNUPLOTpicture{\hss\egroup\egroup}%
\fi
\GNUPLOTunit=0.1bp
{\GNUPLOTspecial{!
%!PS-Adobe-2.0
%%Title: Conclusion/slaterDeterminantNeglectContribution.tex
%%Creator: gnuplot 3.7 patchlevel 3
%%CreationDate: Tue Mar 23 03:09:43 2004
%%DocumentFonts: 
%%BoundingBox: 0 0 360 216
%%Orientation: Landscape
%%Pages: (atend)
%%EndComments
/gnudict 256 dict def
gnudict begin
/Color false def
/Solid false def
/gnulinewidth 5.000 def
/userlinewidth gnulinewidth def
/vshift -33 def
/dl {10 mul} def
/hpt_ 31.5 def
/vpt_ 31.5 def
/hpt hpt_ def
/vpt vpt_ def
/M {moveto} bind def
/L {lineto} bind def
/R {rmoveto} bind def
/V {rlineto} bind def
/vpt2 vpt 2 mul def
/hpt2 hpt 2 mul def
/Lshow { currentpoint stroke M
  0 vshift R show } def
/Rshow { currentpoint stroke M
  dup stringwidth pop neg vshift R show } def
/Cshow { currentpoint stroke M
  dup stringwidth pop -2 div vshift R show } def
/UP { dup vpt_ mul /vpt exch def hpt_ mul /hpt exch def
  /hpt2 hpt 2 mul def /vpt2 vpt 2 mul def } def
/DL { Color {setrgbcolor Solid {pop []} if 0 setdash }
 {pop pop pop Solid {pop []} if 0 setdash} ifelse } def
/BL { stroke userlinewidth 2 mul setlinewidth } def
/AL { stroke userlinewidth 2 div setlinewidth } def
/UL { dup gnulinewidth mul /userlinewidth exch def
      dup 1 lt {pop 1} if 10 mul /udl exch def } def
/PL { stroke userlinewidth setlinewidth } def
/LTb { BL [] 0 0 0 DL } def
/LTa { AL [1 udl mul 2 udl mul] 0 setdash 0 0 0 setrgbcolor } def
/LT0 { PL [] 1 0 0 DL } def
/LT1 { PL [4 dl 2 dl] 0 1 0 DL } def
/LT2 { PL [2 dl 3 dl] 0 0 1 DL } def
/LT3 { PL [1 dl 1.5 dl] 1 0 1 DL } def
/LT4 { PL [5 dl 2 dl 1 dl 2 dl] 0 1 1 DL } def
/LT5 { PL [4 dl 3 dl 1 dl 3 dl] 1 1 0 DL } def
/LT6 { PL [2 dl 2 dl 2 dl 4 dl] 0 0 0 DL } def
/LT7 { PL [2 dl 2 dl 2 dl 2 dl 2 dl 4 dl] 1 0.3 0 DL } def
/LT8 { PL [2 dl 2 dl 2 dl 2 dl 2 dl 2 dl 2 dl 4 dl] 0.5 0.5 0.5 DL } def
/Pnt { stroke [] 0 setdash
   gsave 1 setlinecap M 0 0 V stroke grestore } def
/Dia { stroke [] 0 setdash 2 copy vpt add M
  hpt neg vpt neg V hpt vpt neg V
  hpt vpt V hpt neg vpt V closepath stroke
  Pnt } def
/Pls { stroke [] 0 setdash vpt sub M 0 vpt2 V
  currentpoint stroke M
  hpt neg vpt neg R hpt2 0 V stroke
  } def
/Box { stroke [] 0 setdash 2 copy exch hpt sub exch vpt add M
  0 vpt2 neg V hpt2 0 V 0 vpt2 V
  hpt2 neg 0 V closepath stroke
  Pnt } def
/Crs { stroke [] 0 setdash exch hpt sub exch vpt add M
  hpt2 vpt2 neg V currentpoint stroke M
  hpt2 neg 0 R hpt2 vpt2 V stroke } def
/TriU { stroke [] 0 setdash 2 copy vpt 1.12 mul add M
  hpt neg vpt -1.62 mul V
  hpt 2 mul 0 V
  hpt neg vpt 1.62 mul V closepath stroke
  Pnt  } def
/Star { 2 copy Pls Crs } def
/BoxF { stroke [] 0 setdash exch hpt sub exch vpt add M
  0 vpt2 neg V  hpt2 0 V  0 vpt2 V
  hpt2 neg 0 V  closepath fill } def
/TriUF { stroke [] 0 setdash vpt 1.12 mul add M
  hpt neg vpt -1.62 mul V
  hpt 2 mul 0 V
  hpt neg vpt 1.62 mul V closepath fill } def
/TriD { stroke [] 0 setdash 2 copy vpt 1.12 mul sub M
  hpt neg vpt 1.62 mul V
  hpt 2 mul 0 V
  hpt neg vpt -1.62 mul V closepath stroke
  Pnt  } def
/TriDF { stroke [] 0 setdash vpt 1.12 mul sub M
  hpt neg vpt 1.62 mul V
  hpt 2 mul 0 V
  hpt neg vpt -1.62 mul V closepath fill} def
/DiaF { stroke [] 0 setdash vpt add M
  hpt neg vpt neg V hpt vpt neg V
  hpt vpt V hpt neg vpt V closepath fill } def
/Pent { stroke [] 0 setdash 2 copy gsave
  translate 0 hpt M 4 {72 rotate 0 hpt L} repeat
  closepath stroke grestore Pnt } def
/PentF { stroke [] 0 setdash gsave
  translate 0 hpt M 4 {72 rotate 0 hpt L} repeat
  closepath fill grestore } def
/Circle { stroke [] 0 setdash 2 copy
  hpt 0 360 arc stroke Pnt } def
/CircleF { stroke [] 0 setdash hpt 0 360 arc fill } def
/C0 { BL [] 0 setdash 2 copy moveto vpt 90 450  arc } bind def
/C1 { BL [] 0 setdash 2 copy        moveto
       2 copy  vpt 0 90 arc closepath fill
               vpt 0 360 arc closepath } bind def
/C2 { BL [] 0 setdash 2 copy moveto
       2 copy  vpt 90 180 arc closepath fill
               vpt 0 360 arc closepath } bind def
/C3 { BL [] 0 setdash 2 copy moveto
       2 copy  vpt 0 180 arc closepath fill
               vpt 0 360 arc closepath } bind def
/C4 { BL [] 0 setdash 2 copy moveto
       2 copy  vpt 180 270 arc closepath fill
               vpt 0 360 arc closepath } bind def
/C5 { BL [] 0 setdash 2 copy moveto
       2 copy  vpt 0 90 arc
       2 copy moveto
       2 copy  vpt 180 270 arc closepath fill
               vpt 0 360 arc } bind def
/C6 { BL [] 0 setdash 2 copy moveto
      2 copy  vpt 90 270 arc closepath fill
              vpt 0 360 arc closepath } bind def
/C7 { BL [] 0 setdash 2 copy moveto
      2 copy  vpt 0 270 arc closepath fill
              vpt 0 360 arc closepath } bind def
/C8 { BL [] 0 setdash 2 copy moveto
      2 copy vpt 270 360 arc closepath fill
              vpt 0 360 arc closepath } bind def
/C9 { BL [] 0 setdash 2 copy moveto
      2 copy  vpt 270 450 arc closepath fill
              vpt 0 360 arc closepath } bind def
/C10 { BL [] 0 setdash 2 copy 2 copy moveto vpt 270 360 arc closepath fill
       2 copy moveto
       2 copy vpt 90 180 arc closepath fill
               vpt 0 360 arc closepath } bind def
/C11 { BL [] 0 setdash 2 copy moveto
       2 copy  vpt 0 180 arc closepath fill
       2 copy moveto
       2 copy  vpt 270 360 arc closepath fill
               vpt 0 360 arc closepath } bind def
/C12 { BL [] 0 setdash 2 copy moveto
       2 copy  vpt 180 360 arc closepath fill
               vpt 0 360 arc closepath } bind def
/C13 { BL [] 0 setdash  2 copy moveto
       2 copy  vpt 0 90 arc closepath fill
       2 copy moveto
       2 copy  vpt 180 360 arc closepath fill
               vpt 0 360 arc closepath } bind def
/C14 { BL [] 0 setdash 2 copy moveto
       2 copy  vpt 90 360 arc closepath fill
               vpt 0 360 arc } bind def
/C15 { BL [] 0 setdash 2 copy vpt 0 360 arc closepath fill
               vpt 0 360 arc closepath } bind def
/Rec   { newpath 4 2 roll moveto 1 index 0 rlineto 0 exch rlineto
       neg 0 rlineto closepath } bind def
/Square { dup Rec } bind def
/Bsquare { vpt sub exch vpt sub exch vpt2 Square } bind def
/S0 { BL [] 0 setdash 2 copy moveto 0 vpt rlineto BL Bsquare } bind def
/S1 { BL [] 0 setdash 2 copy vpt Square fill Bsquare } bind def
/S2 { BL [] 0 setdash 2 copy exch vpt sub exch vpt Square fill Bsquare } bind def
/S3 { BL [] 0 setdash 2 copy exch vpt sub exch vpt2 vpt Rec fill Bsquare } bind def
/S4 { BL [] 0 setdash 2 copy exch vpt sub exch vpt sub vpt Square fill Bsquare } bind def
/S5 { BL [] 0 setdash 2 copy 2 copy vpt Square fill
       exch vpt sub exch vpt sub vpt Square fill Bsquare } bind def
/S6 { BL [] 0 setdash 2 copy exch vpt sub exch vpt sub vpt vpt2 Rec fill Bsquare } bind def
/S7 { BL [] 0 setdash 2 copy exch vpt sub exch vpt sub vpt vpt2 Rec fill
       2 copy vpt Square fill
       Bsquare } bind def
/S8 { BL [] 0 setdash 2 copy vpt sub vpt Square fill Bsquare } bind def
/S9 { BL [] 0 setdash 2 copy vpt sub vpt vpt2 Rec fill Bsquare } bind def
/S10 { BL [] 0 setdash 2 copy vpt sub vpt Square fill 2 copy exch vpt sub exch vpt Square fill
       Bsquare } bind def
/S11 { BL [] 0 setdash 2 copy vpt sub vpt Square fill 2 copy exch vpt sub exch vpt2 vpt Rec fill
       Bsquare } bind def
/S12 { BL [] 0 setdash 2 copy exch vpt sub exch vpt sub vpt2 vpt Rec fill Bsquare } bind def
/S13 { BL [] 0 setdash 2 copy exch vpt sub exch vpt sub vpt2 vpt Rec fill
       2 copy vpt Square fill Bsquare } bind def
/S14 { BL [] 0 setdash 2 copy exch vpt sub exch vpt sub vpt2 vpt Rec fill
       2 copy exch vpt sub exch vpt Square fill Bsquare } bind def
/S15 { BL [] 0 setdash 2 copy Bsquare fill Bsquare } bind def
/D0 { gsave translate 45 rotate 0 0 S0 stroke grestore } bind def
/D1 { gsave translate 45 rotate 0 0 S1 stroke grestore } bind def
/D2 { gsave translate 45 rotate 0 0 S2 stroke grestore } bind def
/D3 { gsave translate 45 rotate 0 0 S3 stroke grestore } bind def
/D4 { gsave translate 45 rotate 0 0 S4 stroke grestore } bind def
/D5 { gsave translate 45 rotate 0 0 S5 stroke grestore } bind def
/D6 { gsave translate 45 rotate 0 0 S6 stroke grestore } bind def
/D7 { gsave translate 45 rotate 0 0 S7 stroke grestore } bind def
/D8 { gsave translate 45 rotate 0 0 S8 stroke grestore } bind def
/D9 { gsave translate 45 rotate 0 0 S9 stroke grestore } bind def
/D10 { gsave translate 45 rotate 0 0 S10 stroke grestore } bind def
/D11 { gsave translate 45 rotate 0 0 S11 stroke grestore } bind def
/D12 { gsave translate 45 rotate 0 0 S12 stroke grestore } bind def
/D13 { gsave translate 45 rotate 0 0 S13 stroke grestore } bind def
/D14 { gsave translate 45 rotate 0 0 S14 stroke grestore } bind def
/D15 { gsave translate 45 rotate 0 0 S15 stroke grestore } bind def
/DiaE { stroke [] 0 setdash vpt add M
  hpt neg vpt neg V hpt vpt neg V
  hpt vpt V hpt neg vpt V closepath stroke } def
/BoxE { stroke [] 0 setdash exch hpt sub exch vpt add M
  0 vpt2 neg V hpt2 0 V 0 vpt2 V
  hpt2 neg 0 V closepath stroke } def
/TriUE { stroke [] 0 setdash vpt 1.12 mul add M
  hpt neg vpt -1.62 mul V
  hpt 2 mul 0 V
  hpt neg vpt 1.62 mul V closepath stroke } def
/TriDE { stroke [] 0 setdash vpt 1.12 mul sub M
  hpt neg vpt 1.62 mul V
  hpt 2 mul 0 V
  hpt neg vpt -1.62 mul V closepath stroke } def
/PentE { stroke [] 0 setdash gsave
  translate 0 hpt M 4 {72 rotate 0 hpt L} repeat
  closepath stroke grestore } def
/CircE { stroke [] 0 setdash 
  hpt 0 360 arc stroke } def
/Opaque { gsave closepath 1 setgray fill grestore 0 setgray closepath } def
/DiaW { stroke [] 0 setdash vpt add M
  hpt neg vpt neg V hpt vpt neg V
  hpt vpt V hpt neg vpt V Opaque stroke } def
/BoxW { stroke [] 0 setdash exch hpt sub exch vpt add M
  0 vpt2 neg V hpt2 0 V 0 vpt2 V
  hpt2 neg 0 V Opaque stroke } def
/TriUW { stroke [] 0 setdash vpt 1.12 mul add M
  hpt neg vpt -1.62 mul V
  hpt 2 mul 0 V
  hpt neg vpt 1.62 mul V Opaque stroke } def
/TriDW { stroke [] 0 setdash vpt 1.12 mul sub M
  hpt neg vpt 1.62 mul V
  hpt 2 mul 0 V
  hpt neg vpt -1.62 mul V Opaque stroke } def
/PentW { stroke [] 0 setdash gsave
  translate 0 hpt M 4 {72 rotate 0 hpt L} repeat
  Opaque stroke grestore } def
/CircW { stroke [] 0 setdash 
  hpt 0 360 arc Opaque stroke } def
/BoxFill { gsave Rec 1 setgray fill grestore } def
/Symbol-Oblique /Symbol findfont [1 0 .167 1 0 0] makefont
dup length dict begin {1 index /FID eq {pop pop} {def} ifelse} forall
currentdict end definefont pop
end
%%EndProlog
}}%
\GNUPLOTpicture(3600,2160)
{\GNUPLOTspecial{"
%%Page: 1 1
gnudict begin
gsave
0 0 translate
0.100 0.100 scale
0 setgray
newpath
1.000 UL
LTb
350 300 M
63 0 V
3037 0 R
-63 0 V
350 652 M
63 0 V
3037 0 R
-63 0 V
350 1004 M
63 0 V
3037 0 R
-63 0 V
350 1356 M
63 0 V
3037 0 R
-63 0 V
350 1708 M
63 0 V
3037 0 R
-63 0 V
350 2060 M
63 0 V
3037 0 R
-63 0 V
350 300 M
0 63 V
0 1697 R
0 -63 V
738 300 M
0 63 V
0 1697 R
0 -63 V
1125 300 M
0 63 V
0 1697 R
0 -63 V
1513 300 M
0 63 V
0 1697 R
0 -63 V
1900 300 M
0 63 V
0 1697 R
0 -63 V
2288 300 M
0 63 V
0 1697 R
0 -63 V
2675 300 M
0 63 V
0 1697 R
0 -63 V
3063 300 M
0 63 V
0 1697 R
0 -63 V
3450 300 M
0 63 V
0 1697 R
0 -63 V
1.000 UL
LTb
350 300 M
3100 0 V
0 1760 V
-3100 0 V
350 300 L
1.000 UL
LT0
3087 1947 M
263 0 V
350 2060 M
31 -585 V
32 -391 V
444 824 L
475 650 L
507 533 L
31 -77 V
31 -52 V
32 -35 V
31 -23 V
31 -15 V
31 -10 V
32 -7 V
31 -5 V
31 -3 V
32 -2 V
31 -1 V
31 -1 V
32 -1 V
31 0 V
31 0 V
32 -1 V
31 0 V
31 0 V
32 0 V
31 0 V
31 0 V
31 0 V
32 0 V
31 0 V
31 0 V
32 0 V
31 0 V
31 0 V
32 0 V
31 0 V
31 0 V
32 0 V
31 0 V
31 0 V
32 0 V
31 0 V
31 0 V
31 0 V
32 0 V
31 0 V
31 0 V
32 0 V
31 0 V
31 0 V
32 0 V
31 0 V
31 0 V
32 0 V
31 0 V
31 0 V
32 0 V
31 0 V
31 0 V
31 0 V
32 0 V
31 0 V
31 0 V
32 0 V
31 0 V
31 0 V
32 0 V
31 0 V
31 0 V
32 0 V
31 0 V
31 0 V
32 0 V
31 0 V
31 0 V
31 0 V
32 0 V
31 0 V
31 0 V
32 0 V
31 0 V
31 0 V
32 0 V
31 0 V
31 0 V
32 0 V
31 0 V
31 0 V
32 0 V
31 0 V
31 0 V
31 0 V
32 0 V
31 0 V
31 0 V
32 0 V
31 0 V
31 0 V
32 0 V
31 0 V
1.000 UL
LT1
3087 1847 M
263 0 V
350 300 M
31 3 V
32 6 V
31 8 V
31 10 V
32 10 V
31 9 V
31 9 V
32 8 V
31 6 V
31 6 V
31 4 V
32 3 V
31 2 V
31 1 V
32 1 V
31 -1 V
31 -1 V
32 -2 V
31 -2 V
31 -2 V
32 -3 V
31 -3 V
31 -3 V
32 -4 V
31 -3 V
31 -3 V
31 -4 V
32 -3 V
31 -3 V
31 -4 V
32 -3 V
31 -2 V
31 -3 V
32 -3 V
31 -2 V
31 -3 V
32 -2 V
31 -2 V
31 -2 V
32 -2 V
31 -2 V
31 -1 V
31 -2 V
32 -1 V
31 -1 V
31 -2 V
32 -1 V
31 -1 V
31 -1 V
32 0 V
31 -1 V
31 -1 V
32 -1 V
31 0 V
31 -1 V
32 0 V
31 -1 V
31 0 V
31 0 V
32 -1 V
31 0 V
31 0 V
32 -1 V
31 0 V
31 0 V
32 0 V
31 0 V
31 -1 V
32 0 V
31 0 V
31 0 V
32 0 V
31 0 V
31 0 V
31 0 V
32 0 V
31 0 V
31 -1 V
32 0 V
31 0 V
31 0 V
32 0 V
31 0 V
31 0 V
32 0 V
31 0 V
31 0 V
32 0 V
31 0 V
31 0 V
31 0 V
32 0 V
31 0 V
31 0 V
32 0 V
31 0 V
31 0 V
32 0 V
31 0 V
1.000 UL
LT2
3087 1747 M
263 0 V
350 300 M
31 0 V
32 0 V
31 0 V
31 0 V
32 0 V
31 0 V
31 0 V
32 0 V
31 0 V
31 0 V
31 0 V
32 0 V
31 0 V
31 0 V
32 0 V
31 0 V
31 0 V
32 0 V
31 0 V
31 0 V
32 0 V
31 1 V
31 0 V
32 0 V
31 0 V
31 0 V
31 0 V
32 0 V
31 0 V
31 0 V
32 0 V
31 1 V
31 0 V
32 0 V
31 0 V
31 0 V
32 1 V
31 0 V
31 0 V
32 1 V
31 0 V
31 0 V
31 1 V
32 0 V
31 1 V
31 0 V
32 1 V
31 1 V
31 1 V
32 0 V
31 1 V
31 1 V
32 1 V
31 2 V
31 1 V
32 1 V
31 2 V
31 1 V
31 2 V
32 2 V
31 2 V
31 2 V
32 2 V
31 3 V
31 2 V
32 3 V
31 3 V
31 2 V
32 3 V
31 4 V
31 3 V
32 3 V
31 4 V
31 3 V
31 3 V
32 4 V
31 3 V
31 3 V
32 3 V
31 2 V
31 2 V
32 2 V
31 1 V
31 1 V
32 -1 V
31 -1 V
31 -2 V
32 -3 V
31 -4 V
31 -6 V
31 -6 V
32 -8 V
31 -9 V
31 -9 V
32 -10 V
31 -10 V
31 -8 V
32 -6 V
31 -3 V
1.000 UL
LT3
3087 1647 M
263 0 V
350 300 M
31 0 V
32 0 V
31 0 V
31 0 V
32 0 V
31 0 V
31 0 V
32 0 V
31 0 V
31 0 V
31 0 V
32 0 V
31 0 V
31 0 V
32 0 V
31 0 V
31 0 V
32 0 V
31 0 V
31 0 V
32 0 V
31 0 V
31 0 V
32 0 V
31 0 V
31 0 V
31 0 V
32 0 V
31 0 V
31 0 V
32 0 V
31 0 V
31 0 V
32 0 V
31 0 V
31 0 V
32 0 V
31 0 V
31 0 V
32 0 V
31 0 V
31 0 V
31 0 V
32 0 V
31 0 V
31 0 V
32 0 V
31 0 V
31 0 V
32 0 V
31 0 V
31 0 V
32 0 V
31 0 V
31 0 V
32 0 V
31 0 V
31 0 V
31 0 V
32 0 V
31 0 V
31 0 V
32 0 V
31 0 V
31 0 V
32 0 V
31 0 V
31 0 V
32 0 V
31 0 V
31 0 V
32 0 V
31 0 V
31 0 V
31 0 V
32 0 V
31 0 V
31 0 V
32 1 V
31 0 V
31 0 V
32 1 V
31 1 V
31 1 V
32 2 V
31 3 V
31 5 V
32 7 V
31 10 V
31 15 V
31 23 V
32 35 V
31 52 V
31 77 V
32 117 V
31 174 V
31 260 V
32 391 V
31 585 V
stroke
grestore
end
showpage
}}%
\put(3037,1647){\rjust{$\Psi_{1s}^2(r)$}}%
\put(3037,1747){\rjust{$\Psi_{3d}^2(r)$}}%
\put(3037,1847){\rjust{$\Psi_{3d}^1(r)$}}%
\put(3037,1947){\rjust{$\Psi_{1s}^1(r)$}}%
\put(3372,476){\ljust{D}}%
\put(2907,476){\ljust{C}}%
\put(892,476){\ljust{B}}%
\put(428,476){\ljust{A}}%
\put(3372,476){\ljust{D}}%
\put(2907,476){\ljust{C}}%
\put(892,476){\ljust{B}}%
\put(428,476){\ljust{A}}%
\put(1900,50){\cjust{r}}%
\put(3450,200){\cjust{ 4}}%
\put(3063,200){\cjust{ 3.5}}%
\put(2675,200){\cjust{ 3}}%
\put(2288,200){\cjust{ 2.5}}%
\put(1900,200){\cjust{ 2}}%
\put(1513,200){\cjust{ 1.5}}%
\put(1125,200){\cjust{ 1}}%
\put(738,200){\cjust{ 0.5}}%
\put(350,200){\cjust{ 0}}%
\put(300,2060){\rjust{ 1}}%
\put(300,1708){\rjust{ 0.8}}%
\put(300,1356){\rjust{ 0.6}}%
\put(300,1004){\rjust{ 0.4}}%
\put(300,652){\rjust{ 0.2}}%
\put(300,300){\rjust{ 0}}%
\endGNUPLOTpicture
\endgroup
\endinput

%  \caption{Plot of two selected hydrogen radial solutions; $\Psi_{1s}$
%  and $\Psi_{3d}$, for the nucleus charge $Z=10$.
%  }
%  \label{autoCorrelationTimeScaleIllustration}
%\end{center}
%\end{figure}

