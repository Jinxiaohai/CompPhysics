\documentstyle[a4wide]{article}
\newcommand{\OP}[1]{{\bf\widehat{#1}}}

\newcommand{\be}{\begin{equation}}

\newcommand{\ee}{\end{equation}}

\begin{document}

\pagestyle{plain}

\section*{Thesis title: Coupled-Cluster theory and density functional theory studies of quantum dots}

The aim of this thesis is to study the structure of quantum dots using
Coupled cluster theory, combining results
from Hartree-Fock calculations in order to achieve a as good as possible
variational wave function. The thesis will explore various Coupled cluster
approaches and use these to define the best possible density functional.
Semiconductor quantum dots are structures where
charge carriers are confined in all three spatial dimensions, 
the dot size being of the order of the Fermi wavelength 
in the host material, typically between  10 nm and  1 $\mu$m.
The confinement is usually achieved by electrical gating of a 
two-dimensional electron gas (2DEG), 
possibly combined with etching techniques. Precise control of the
number of electrons in the conduction band of a quantum dot 
(starting from zero) has been achieved in GaAs heterostructures. 
The electronic spectrum of typical quantum dots
can vary strongly when an external magnetic field is applied, 
since the magnetic length corresponding to typical 
laboratory fields  is comparable to typical dot sizes.
In coupled quantum dots Coulomb blockade effects, 
tunneling between neighboring dots, and magnetization 
have been observed as well as the formation of a
delocalized single-particle state. 

Of particular interest for this thesis is the application of quantum dots to solar cells, as they can increase the efficieny of such cells. If quantum dots are grown within a second and wider band of semiconductor material, then the quantum dot introduces a discrete level within the band gap of the host semiconductor.
Different types of quantum dots confinements will be tested in this thesis, from
harmonic oscillator traps to square well traps. 
Alternatively, one can also study the quantum hall effect in the strongly interacting regime.


This thesis 
entails the development of a coupled-cluster code which includes so-called triples corrections, based on our existing codes in order
programs to solve Schr\"odingers equation
and obtain various expectation values of interest, such as the energy
of the ground state and excited states. 
Being an {\em ab initio} method, these calculations will provide a as best as possible detrmination of the 
ground state wave function. This wave function will in turn be used to define the quantum
mechanical density.  The density will be used to construct a density functional for quantum dots
using the adiabatic-connection method as described by Teale {\em et al} in J.~Chem.~Phys.~{\bf 130},
104111 (2009).  The results will be compared with existing density functional for quantum dots.

The results are expected to be published in leading journals.




A progress plan for this thesis project is given at the end.


\section*{Progress plan and milestones}
The aims and progress plan of this thesis are as follows
\begin{itemize}
\item Fall 2010: Develop first a Hartree-Fock code for electrons trapped in a single harmonic oscillator  
in two dimensions.   This part entails developing a code for computing the Coulomb interaction
in two dimensions in the laboratory system.
\item The Hartree-Fock interaction is then used as input to our exisiting coupled cluster codes which include excitations up to two-particle-two-hole excitations.
The results will be compared with results from Monte Carlo 
techniques for 2, 6, 12 and 20 
electrons in a single harmonic oscillator well.
\item Fall 2010: Write a code which includes three-particle-three-hole excitations in the coupled-cluster machinery. Parallelization of the coupled-cluster codes is necessary in order to study up to approximately 100 electrons.

 \item Spring 2011: The obtained ground states will in turn be used to define a as exact as possible 
density functional for quantum dots
using the adiabatic-connection method. The density functional can in turn be used to model
systems with a large number of elctrons in quantum dots. Comparisons with a density functional derived from Monte Carlo  methods will also be made in order to
test the validity of the Coupled-Cluster codes with triples corrections.
\end{itemize}
 


The thesis is expected to be handed in June 1 2011.

\begin{thebibliography}{999}

\bibitem {ref1}  A.~M.~Teale, S.~Coriani, and T.~Helgaker, J.~Chem.~Phys.~{\bf 130},
104111 (2009).
\bibitem{ref2} Jenny Olsen, {\em The Physics of Solar Cells} (Imperial College, Press, London,  2009).


\end{thebibliography}



\end{document}



