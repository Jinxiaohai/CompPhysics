\chapter{Hatree-Fock Method}
%In this chapter we shall...................
%Hartree-Fock theory and DFT are simplifications of the manybody problem with moving particles in a potential field. Physical systems such as atoms, molecules and  nuclei. In such systems particles are not only affected by the main potential set up by a \emph{core}, but aswell as the field generated by the other particles. 

%The first approximation is to assume that the kinetic energy of the \emph{core} is negligible compared to the particles surrounding it. Then we just have to solve the Schr\"{o}dinger equation for particles instead of the whole system, particles$+$core. This is the philosophy of \emph{Born-Oppenheimer approximation}.

The exact solution to the Schr\"{o}inger equation cannot be obtained in most of the systems of chemical interest (see \cite{helgaker}). The number of particles involved in those problems are just to big for computers to solve. Therefore approximations are needed. The approximations would of course depend on the physical system. We could just try some potential and interactions that works. But the many-body theories have hierachcal structure when it comes to approximations. The advantages of this is the fact that we build up experience about the given models. And we can do benchmarks and comparison tests. This allows too differentiate between different type of \emph{effects} in the theories. In this chapter we are going to look into one of the first approximations, the Hartree-Fock method.

\section{Introduction}
\label{sec:Hatree-Fock Method:Introduction}
The independent particle picture is an assumption that the electron-electron interaction is rather weak. And each electron could be viewed as independent particles which sees an effective field set up by the other electrons. This is the philosophies of Hartree-Fock theories and Density functional theories. 

The Hartree-Fock method is an optimization problem using Lagrange multipliers as the mathematical tool. The integral we want to minimize are in general

\begin{equation}
E[\Phi] = \int_a^b f(\Phi(r),\frac{\partial \Phi}{\partial r}, r)d^3 r
 \label{def:integral we want to minimize}
\end{equation}
%
We want to find a function $\Phi$ that minimizes the \emph{functional} $E[\Phi]$. The set of functions $V$ have the following conditions: $V = \ql \{ \Phi:[-\infty,\infty] \rightarrow \mathbb{R}:\text{$\Phi$ is continous and differentiable}: \braket{\Phi}{\Phi} = 1 \qr \}$. The unkowns are the functions in $V,\frac{\partial \Phi}{\partial r}$ and $r$. Although the integral limits $a,b$ are defined, the integration path is not. We want to find a path with a given set of unknowns such that $\delta E = 0$. This will give us minima, maxima or saddle points. So we have check if we have a minima after finding the solution.

The quantum mechanical functional we want to minimize is

\begin{equation}
E[\Phi] = \frac{\bra{\Phi}\hat{H}\ket{\Phi}}{\braket{\Phi}{\Phi}} =  \frac{\int \Phi^* \hat{H} \Phi d\tau}{\int \Phi^* \Phi d \tau}
 \label{def:energyfunctional}
\end{equation}
If $\Phi$ is an eigenstate of the $\hat{H}$, then the functional will be stationary
\begin{equation}
\hat{H} \ket{\Phi} = E_0 \ket{\Phi} \Rightarrow E[\Phi] = E_0 \Rightarrow \delta E[\Phi] = 0 
 \label{def:eigenstate then stationary}
\end{equation}
%
Conversely we can show a stationary point is an eigenstates of the Hamiltonian (\ref{def:eigenstate then stationary}), and therefore a solutions of the Schr\"{o}dinger equation. 
%
\begin{proof}
From (\ref{def:energyfunctional}) and the fact that setting $\delta E[\Phi]$ = 0 gives us
\begin{align}
0 &= \delta [ \bra{\Phi}\hat{H}\ket{\Phi} - E[\Phi] \ql (\braket{\Phi}{\Phi} - 1 \qr ) ] \nonumber \\
0 &= \delta [ \bra{\Phi} \ql( \hat{H}-E[\Phi] \qr ) \ket{\Phi}+ E[\Phi] ] \nonumber \\
0 &=  \bra{\delta \Phi} \ql( \hat{H}-E[\Phi] \qr ) \ket{\Phi}+  \bra{\Phi} \ql( \hat{H}-E[\Phi] \qr ) \ket{\delta \Phi}+ \delta E[\Phi] \nonumber \\
0 &=  \bra{\delta \Phi} \ql( \hat{H}-E[\Phi] \qr ) \ket{\Phi}+  \bra{\Phi} \ql( \hat{H}-E[\Phi] \qr ) \ket{\delta \Phi} 
\label{proof:stationary points are eigenvalues}
\end{align}
We can add a phase $\delta \Phi \rightarrow i\delta \Phi$. Adding a phase to a wave function will not change the expectation value. Then we get two se of equations 
\begin{align}
I. \qquad & \bra{\delta \Phi} \ql( \hat{H}-E \qr ) \ket{\Phi}+  \bra{\Phi} \ql( \hat{H}-E \qr ) \ket{\delta \Phi}  = 0\\
II. \qquad& \bra{\delta \Phi} \ql( \hat{H}-E \qr ) \ket{\Phi}-  \bra{\Phi} \ql( \hat{H}-E \qr ) \ket{\delta \Phi} = 0
 \label{proof:arbritrary phase}
\end{align}
We have multiplied with $i$ in (II) and remember that $i \ket{\delta \Phi} = -i\bra{\delta \Phi}$. And adding them together
\begin{equation}
\bra{\delta \Phi} \ql( \hat{H}-E \qr ) \ket{\Phi} = 0
 \label{proof:adding I with II}
\end{equation}
The choice of $\bra{\delta \Phi}$ is arbritrary, so the equation must hold for all possible $\bra{\delta \Phi}$'s. Therefore this can only be satisfied if $\Phi$ is an eigenfunction to $\hat{H}$.
\end{proof}
%
\noindent This is equivalent to Rayleigh-Ritz principle \cite{arfken} that tells us that the functional $\Omega_{RR} = \bra{\Psi}\hat{H}-E\ket{\Psi}$  is stationary at any eigenfunctions $H\ket{\Phi_m} = E_m \ket{\Phi_m}$. 
%
One important feature of the functional (\ref{def:energyfunctional}) is that errors are in the second order in $\delta \Phi$. 
 
\begin{proof}
Our trial wavefunction is $\ket{\phi} = \ket{\psi_0} + \lambda\ket{\psi}$. $\ket{\psi_0}$ is a stationary state, $\ket{\psi}$ is our guess, and $\lambda$ is complex number. Using the relations
\begin{align}
(\hat{H}-E_0) \ket{\psi_0} &= 0 \nonumber \\
(\hat{H}-E_0) [\ket{\phi} - \lambda \ket{\psi} ] &= 0 \nonumber \\
(\hat{H}-E_0) \ket{\phi} &= \lambda (\hat{H}-E_0 ) \ket{\psi} 
 \label{proof:need relations}
\end{align}
%
We get
\begin{align}
E[\Phi] - E_0 & = \frac{\bra{\phi}(\hat{H}-E_0)\ket{\phi}}{\braket{\phi}{\phi}} \nonumber \\ 
&= \frac{\bra{\lambda\psi}(\hat{H}-E_0)\ket{\lambda\psi}}{\braket{\phi}{\phi}} \nonumber \\
&= |\lambda|^2 \frac{\bra{\psi}(\hat{H}-E_0)\ket{\psi}}{\braket{\phi}{\phi}}
\end{align}
\end{proof}

\section{Derivation of the Hartree-Fock equations}
The HF ansatz reads

\begin{equation}
\Phi_{\text{HF}} = \ket{pqrs...}
 \label{def:Hartree Fock ansatz}
\end{equation}
%
Where $pqrs...$ are the HF spin-orbitals. $\Phi_{\text{HF}}$ is a SD with $N$ states written in second quantization formalism. We could vary these directly, giving us the Hartree-Fock-Roothaan method \cite{roothaan}. Another possibility is to expand our spin-orbitals $\ket{a}$ as a linear combination of a finite number of basis states $\ket{\alpha}$. (A unitary transformation of the states $\ket{a}$?) The sum goes to infinite in general but we do a truncation. (Why does this work? with the truncation).
\begin{equation}
 \ket{a} = \sum_{\alpha}^n C_{a\alpha} \ket{\alpha}
 \label{def:spin-orbital expantion}
\end{equation}
%
$C_{a\alpha}$ is the expansion coefficient of an unitary matrix $C^{\dagger}C = 1$. Writing the Hamiltonian in this basis up to two-body interactions gives

\begin{equation}
\hat{H} = \sum_{ab} \bra{a}h\ket{b}a_a^{\dagger}a_b + \frac{1}{2} \sum_{abcd} \bra{ab}v\ket{cd} a_a^{\dagger}a_b^{\dagger} a_d a_c
 \label{def:Hamiltonian in second quantization}
\end{equation}
Note that the Hamiltionian is in second quantization form (see section \ref{sec:second quantization} and \cite{bartlett} for details). $h$ is just the one-body operator, and $v$ is the two-body operator. 
%
We will restrict our system to a closed shell system (RHF-equations, see cite...), i.e. all possible spin-orbital levels are filled. Our trial wavefunction $\ket{\Phi_{\text{HF}}}$ will be the \emph{Fermi vacuum} $\ket{\bf0}$ (see section \ref{sec:particle-hole formalism}). We will use Wick's theorem to evaluate $\bra{\bf 0}\hat{H}\ket{\bf 0}$. First the Hamiltonian (\ref{def:Hamiltonian in second quantization}) have to be rewritten in the quasi-operator representation
\begin{equation}
\hat{H} = \sum_{ij} \bra{i}h\ket{j}b_ib_j^{\dagger} + \frac{1}{2} \sum_{ijkl} \bra{ij}v\ket{kl} b_ib_j b_l^{\dagger} b_k^{\dagger}
 \label{def:Hamiltonian in quasi-operator representation}
\end{equation}
Wick's theorem on quasi-operators
\begin{align}
\bra{\bf 0} b_{i}b_{j}^{\dagger} \ket{\bf 0} &= \delta_{ij} \qquad &\text{from (\ref{eq:vacuumexpectation2redo one-body})} \\
\bra{\bf 0} b_ib_j b_l^{\dagger} b_k^{\dagger} \ket{\bf 0} &= \delta_{kj}\delta_{li} - \delta_{lj}\delta_{ki} \qquad &\text{from (\ref{eq:vacuumexpectation2redo})} 
 \label{eq:evaluate quasi-operators}
\end{align}
Lead us to the following energy functional
\begin{align}
E[\Phi_{\text{HF}}] &= \bra{\bf 0}\hat{H}\ket{\bf 0} = \sum_{i} \bra{i}h\ket{i} + \frac{1}{2}   \sum_{ijkl} [\bra{ij}v\ket{ij} - \bra{ij}v\ket{ji}] \nonumber \\
        &= \sum_{i} \bra{i}h\ket{i} + \frac{1}{2}   \sum_{ij} \bra{ij}v\ket{ij}_{\text{AS}}
 \label{eq:energyfunctional}
\end{align}
%
Inserting the new basis states (\ref{def:spin-orbital expantion}) in this expression yields 
%
\begin{equation}
E[\Phi_{\text{HF}}] = \sum_{a}^N \sum_{\alpha \beta}^n C_{a\alpha}^*C_{b\beta} \bra{\alpha}h\ket{\beta} + \frac{1}{2} \sum_{ab}^N\sum_{\alpha\beta\gamma\delta}^n C_{a\alpha}^*C_{b\beta}^* C_{a\gamma}C_{b\delta} \bra{\alpha\beta}v\ket{\gamma\delta}_{\text{AS}}
 \label{eq:energyfunctional2}
\end{equation}
%
The second-quantized form of an operator is unchanged under a unitary transformation of the basis (see \cite{bartlett}). 

As mention the method of choice for minimizing $E[\Phi_{HF}]$ is the Lagrange multipliers method (see \cite{boas}).

In general we want to find a stationary point $p=(x_1,x_2,...,x_n)$ of a function $f(p)$ with multiple constraints
\begin{align}
g_1(p) &= 0 \nonumber \\
g_2(p) &= 0 \nonumber \\
\vdots  \\
g_N(p) &= 0  \nonumber 
 \label{def:constraints}
\end{align}
%
Then a point $p$ is a stationary if and only if 
%
\begin{equation}
\nabla_{\nu} f(p) - \sum_{i=1}^N \lambda_i \nabla_\nu g_i(p) = 0, \qquad \forall \nu \in p
 \label{eq:langrange multiplier}
\end{equation}
%
$\lambda_i$ is the Langrangian multipliers. We get a system of $N\cdot n$ equations to be solved for the $(N+n)$-set of variables  $\lambda_1,\lambda_2...\lambda_N,x_1,x_2,...x_n$

In our case the constraint is the HF spin-orbitals are orthonormal (Because we have done a unitary transformation?)(which they have to be? No Since a og b can have same spin, depends on the physical system)

\begin{equation}
\braket{a}{b} = \sum_{\alpha}^n C_{a\alpha}^*C_{b\alpha} = \delta_{ab}
 \label{def: Hartree Fock constraint}
\end{equation}
%
We want to find the stationary point of the energy functional (\ref{eq:energyfunctional2}) with respect to a specific coefficient $C_{k\mu}^*$ in (\ref{def:spin-orbital expantion}). We dont have to do the same for $C_{k\mu}$ since they only differ by a phase. This means that for our HF spin-orbitals $\ket{a} \rightarrow i\ket{a}$ will not change the expectation value. 

We define
\begin{align}
F &\equiv E[\Phi_{\text{HF}}] - \sum_a^N \lambda_a g_a \nonumber \\
  &= E[\Phi_{\text{HF}}] - \sum_a^N \lambda_a \sum_\alpha^n C_{a\alpha}^*C_{a\alpha}
 \label{def:Big F}
\end{align}
%
Then we take the partial derivate of $F$ with respect to $C_{k\mu}^*$ using Eq. (\ref{eq:langrange multiplier}), yielding

\begin{align}
\frac{\partial}{\partial C_{k\mu}^*}  &\ql(  \sum_{a}^N\qr.  \sum_{\alpha \beta}^n C_{a\alpha}^*C_{b\beta} \bra{\alpha}h\ket{\beta} + \frac{1}{2} \sum_{ab}^N\sum_{\alpha\beta\gamma\delta}^n C_{a\alpha}^*C_{b\beta}^*  C_{a\gamma}C_{b\delta} \bra{\alpha\beta}v\ket{\gamma\delta}_{\text{AS}} \nonumber - \ql. \sum_a^N \lambda_a \sum_\alpha C_{a\alpha}^*C_{a\alpha}  \qr ) \\
 &= \sum_{\alpha \beta}^n C_{k\beta} \bra{\alpha}h\ket{\beta} + \frac{1}{2} \sum_{a}^N \sum_{\alpha\beta\gamma\delta}^n C_{a\beta}^*C_{k\gamma}  C_{a\delta} \bra{\alpha\beta}v\ket{\gamma\delta}_{\text{AS}} - \lambda_k \sum_\alpha^n C_{k\alpha} \\
&= \sum_{\alpha}^n\ql (\sum_\beta^n C_{k\beta} \bra{\alpha}h\ket{\beta} + \frac{1}{2} \sum_{a}^N \sum_{\beta\gamma\delta}^n C_{a\beta}^*C_{k\gamma}  C_{a\delta} \bra{\alpha\beta}v\ket{\gamma\delta}_{\text{AS}} - \lambda_k C_{k\alpha} \qr ) = 0  \label{eq:partial of F} 
\end{align}
%
The factor $1/2$ dissapears because of the product rule when we derivate $C_{a\alpha}^*C_{b\beta}^*  C_{a\gamma}C_{b\delta}$, we get two cases, when $\{a = k, \alpha = \mu \}$ and $\{b = k, \beta = \mu\}$. We rewrite Eq. (\ref{eq:partial of F})

\begin{equation}
\sum_\gamma^n C_{k\gamma} \bra{\alpha}h\ket{\gamma} + \frac{1}{2} \sum_{a}^N \sum_{\beta\gamma\delta}^n C_{a\beta}^*C_{k\gamma}  C_{a\delta} \bra{\alpha\beta}v\ket{\gamma\delta}_{\text{AS}}  = \lambda_k C_{k\alpha}
  \label{eq: rewritten equation} 
\end{equation}
Where we have changed the summation index $\beta \rightarrow \gamma$ in the first sum. And this than be simplified further

\begin{equation}
\sum_\gamma^n \ql ( \bra{\alpha}h\ket{\gamma} + \frac{1}{2} \sum_{a}^N \sum_{\beta\delta}^n C_{a\beta}^*  C_{a\delta} \bra{\alpha\beta}v\ket{\gamma\delta}_{\text{AS}} \qr ) C_{k\gamma} = \lambda_k C_{k\alpha}
 \label{def: simplified HF equation}
\end{equation}
%
By defining the Hartree-Fock Hamiltonian as 
\begin{equation}
h_{\alpha \gamma}^{\text{HF}} \equiv \bra{\alpha}h\ket{\gamma} + \frac{1}{2} \sum_{a}^N \sum_{\beta\delta}^n C_{a\beta}^*  C_{a\delta} \bra{\alpha\beta}v\ket{\gamma\delta}_{\text{AS}}
 \label{def: HF Hamiltonian}
\end{equation}
We finally obtain the Hartree-Fock equation
\begin{equation}
\sum_\gamma^n h_{\alpha\gamma}^{\text{HF}} C_{k\gamma} = \lambda_k C_{k\alpha}
 \label{def: The final HF equation}
\end{equation}
%
This is a non-linear set of equations, meaning that the set of equations changes with solutions we find. One way to solve this is by the \emph{self-consisten field method}. 



%
%\emph{the Hylleraas-Undheim theorem} \cite{hylleraas} and \cite{helgaker}
%
%More Hartree-Fock Approximation (see Heinnonen 59 kapt 7) And about Restricted Hartree-Fock approximation and the symmetry dilemma.
%
%The quasi-particle concept Heinnonen 313 kapt 26
%
%Time-independent perturbation theory Heinnonen 197 kapt 17
%
%Particle and hole operators and Wick's theorem kapt 19,20,21,22...
\section{Self-Consistent Field Method}
This is an iterative method that involves diagonalizing the $h_{\alpha\beta}^{\text{HF}}$-matrix for each iteration and summing the eigenvalues. It reaches selfconsitensy  

In a quantum mechanical system of many particles, particles are interacting with each other and there is \emph{correlation} between them. Therefore we have to find a wavefunction which describe the system as an whole. Our guess is that this wavefunction could have the form of an Slater-Determinant which is product of the single-particle wavefunctions, which depends on both spin and positional coordinates. 

The single-particle wavefunctions must be \emph{self-consistent}, i.e. they are solutions of the  Schr\"{o}dinger equations for the average field produced by the other particles, and they determine the average potential of the field.

One of the earliest methods relating to a self-consistent field was the Thomas-Fermi method (L. Thomas and E. Fermi) (ref). It did not use wave functions but the density distribution of the particles. The self-consistent Thomas-Fermi field explains the sequence in which electron shells are filled in atoms. 

The Hartree method does not take into account the PEP, an improved self-consistent field was given by the Hartree-Fock method, which was introduced by V.A. Fock in 1930 (ref). He used the SD instead of just the product of single-particle wavefunctions. We use this method to find the single-particle states that gives us minimum in energy.

Hartree-Fock method gives contribution for pairs of particles with opposite spins, but not the case of when the pairs of the spins are parallel? (Attraction?). A generalization of the Hartree-Fock method that take this into account aswell was developed by N. N. Bogoliubov in 1958 (ref?). He used this in the theory of superconductivity and the theory of heavy nuclei.

Historically, the concept of a self-consistent field was introduced in 1907 by the French physiscist P. Weiss \cite{weiss}. He found that the magnetic moment of each atom in a ferromagnetic material is proportional to magnetic moment and is thus \emph{self-consistent}. %(Vi har en partikkel som ser et gjennomsnittlig felt dannet av andre partikler, men som er med p� � skape dette feltet.)

The interaction, averaged in a certain way, between a given particle and the other particles of a quantum-mechanical system consisting of many particles. Because the problem of many interacting particles is very complex and has no exact solution, calculations are done by approximate methods. One of the most often used approximate methods of quantum mechanics is based on the introduction of a self-consistent field, which permits the many-particle problem to be reduced to the problem of a single particle moving in the average self-consistent field produced by the other particles. 

A concept used to find approximate solutions to the many-body problem in quantum mechanics. The procedure starts with an approximate solution for a particle moving in a single-particle potential, which derives from its average interaction with all the other particles. This average interaction is determined by the wave functions of all the other particles. The equation describing this average interaction is solved and the improved solution obtained is used in the calculation of the interaction term. This procedure is repeated for wave functions until the wave functions and associated energies are not significantly changed in the cycle, self-consistency having been attained. In atomic theory the Hartree-Fock procedure makes use of self-consistent fields. 

We would refer to this as the restricted Hatree-Fock (RHF) method. Which is used in calculations of a \emph{closed-shell} system, i.e. all the levels are filled with electrons. For \emph{open-shell} systems unrestricted Hartree-Fock (UHF) is the method of choice (see \cite{uhf}).

We would like to know how well the HF method preform. Therefore we introduce the term \emph{correlation energy}. 

\begin{equation}
E_{corr} = E_{exact} - E_{HF}
 \label{def:correlation energy}
\end{equation}

It is the difference between the exact nonrelativistic energy (i.e. the FCI energy) and the Hartree-Focl energy. This gives us some indication of how much the electron-electron interaction contributes to our system.

\begin{figure}
\centering
  \scalebox{0.75}{
    \input{electron-correlation.pdftex_t}
}
\caption{A schematic overview over different energies for different approximations}
 \label{fig:Electron correlation energy}
\end{figure}


\section{Outline of the Algorithm}

\begin{figure}
\centering
\includegraphics{HF-flowchart.pdf}
\caption{Flowchart of Hartree-Fock Algorithm. Self-consistency is the condition that $E[\Phi_{new}] - E[\Phi_{old}] = 0$ }
 \label{fig:HF flow chart}
\end{figure}


%%%%%%%%%%%%%%%%%%%%%%%%%%%%%%%%%%%%%%%%%%% Comments %%%%%%%%%%%%%%%%%%%%%%%%%%%%%%%%%%%%%%

% Thjiissen:
% Hartree-Fock method a variational method? We have a SD.
% Individual one-electron wave functions are called \emph{orbitals}.
% The Schr\"{o}dinger equation have a potential determined by the electrons. A coupling between the orbitals via the potential makes the equation (SL?) nonlinear in the orbitals?. 
% 
% The solution to this \emph{nonlinear in the orbitals} is found iteratively in a self-consistency procedure??? 
% 
% The Hartree-Fock method is close the mean-field approach in Statmech???
% 
% Hartree-Fock method is very popular among chemists, and it has also been applied to solids. [See referanser...]
% 
% Hartree-Fock method used in fermionic systems, 
% 
% The Coulomb repulsion between the electrons is represented in an averaged way. 
% The Pauli Principle is included in the HF-theory.
% 
% Even with the position of the nuclei kept fixed (Born-Oppenheimer approximation). But still we have problems with too many degrees of freedom. Especially the the interaction term makes the problem difficult. (In what way?)
% 
% One common method is to ''uncouple`` the problem in which the interaction of one electron with the remaining ones is incorporated in an averaged way into a potential felt by the electron 
% 
% Def: Spin-orbitals are functions depending on the spatial and spin coordinates of one electron (page 48)
% 
% However the Hamiltonian does not act on the spin-dependent part of the spin-orbital (a quantum state |alpha>). Then spatial and the spin part can be seperated into two products. 
% 
% Hartree potential/Hartree term is something unphysical: it contains a coupling between orbital \emph{k} and itself.
% 
% Hartree derived his equation 4.11 in 1927, it neglects exchange as well as other correlations (which?)
% 
% Fock took the antisymmetry into account, exchange term. 
% 
% It is clear that the nonlinear Hartree-Fock equation, must be solved by a selfconsistency iterative procedure. (Describe) This means that we start with a finite number of states N. A guess. These are used to generate a new Fock operator which is diagonalized again and we get new states. This procedure is repeated until the wanted convergence is achieved.
% 
% CI (Configuration Interaction) is a systematic way of improving upon Hartree-Fock theory.
% 
% The exchange term vanished for orthogonal states, pair of opposite spin do not feel this term.
% 
% The derivation of the Hartree-Fock equation is based on a varitaional calculation for the Schr\"{o}dinger equation. 
% 
% We can use Koopman's theorem to approximate excitation energies from a ground state calculation.
% 
% In closed shells systems, all the levels are occupied by two electrons with opposite spin, but in open-shell system there some levels that are partially filled. 
% 
% Calculation with spin-orbitals paired are called restricted Hartree-Fock (RHF). Those in which all spin-orbitals are allowed to have a different spatial dependence are called unrestricted Hartree-Fock (UHF).
% 
% Magnus: \cite{lohne}  The foundation of HF is the variational principle of Rayleigh and Ritz (RR) (See E.K.U Gross)
% 
% Derived the Hartree Fock Equations
% 
% Patrick: \cite{merlot}
% Quantum Dots can be seen as an many-particle problem. (Show how?)
% 
% Hartree-Fock method is minimazation problem based on a mathematical technique known as the Lagrange multipliers, we want to minimize the energy.