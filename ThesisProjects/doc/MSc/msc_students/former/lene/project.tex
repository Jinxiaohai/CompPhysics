\documentstyle[a4wide]{article}
\newcommand{\OP}[1]{{\bf\widehat{#1}}}

\newcommand{\be}{\begin{equation}}

\newcommand{\ee}{\end{equation}}

\begin{document}

\pagestyle{plain}

\section*{Thesis title: Finite element  studies of quantum dots}

{\bf The aim of this thesis is to study numerically systems consisting of two
interacting electrons in two dimensions}, confined to small regions
between layers of semiconductors. 

The thesis deals with the study and developments of stable
numerical approaches to Schr\"odinger's equation for systems of one and
two electrons confined in space, under the action of external time-independent and/or time-dependent one-body interactions, and/or
two-particle interactions. 


These electron systems
are dubbed quantum dots in the literature. 
Semiconductor quantum dots are structures where
charge carriers are confined in all three spatial dimensions, 
the dot size being of the order of the Fermi wavelength 
in the host material, typically between  10 nm and  1 $\mu$m.
The confinement is usually achieved by electrical gating of a 
two-dimensional electron gas.
possibly combined with etching techniques. Precise control of the
number of electrons in the conduction band of a quantum dot 
(starting from zero) has been achieved in GaAs heterostructures. 
The electronic spectrum of typical quantum dots
can vary strongly when an external magnetic field is applied, 
since the magnetic length corresponding to typical 
laboratory fields  is comparable to typical dot sizes.
In coupled quantum dots Coulomb blockade effects, 
tunneling between neighboring dots, and magnetization 
have been observed as well as the formation of a
delocalized single-particle state. 

Quantum dots have been used to fabricate  quantum gates and are also used in the emerging field
of quantum nano medicine.  

More specifically, this thesis aims at a study of the finite element method and finite difference
method of Schr\"odinger's equation for two electrons confined to two dimensions.
It starts with a study of a similar system of one electron.

The thesis can be split into the following tasks:
\begin{itemize}
\item Set up of the one-electron eigenvalue problem with no time dependence using a harmonic 
oscillator interaction which traps the electron. Compare finite difference with possible finite element approaches. The harmonic oscillator potential can be replaced by an attractive Coulomb interaction
and one can simulate a two-dimensional hydrogen atom. Make a critical evaluation of the two
methods.  A rectangular mesh is a most likely a good starting point.
\item Study parallelization  of both methods for the one-electron case.
\item Develop a finite element code and finite difference code for the two-electrons in one harmonic oscillator well, with and without a repulsive Coulomb interaction in two dimensions. Compare with exact results and other methods like configuration interaction methods (available literature and codes at UiO). 
Here it will be important to transform the two-particle equation in the laboratory system to an equation in the relative and center-of-mass coordinates since the interaction depends only on the relative distance betweem the two electrons.

\item The two-electron case should also be parallelized.  

\item 
A confining potential of great interest would be a double harmonic oscillator well.  If time allows,
calculations with such a potential would be very interesting as this confining potential has been used as model for constructing quantum gates based on quantum dots.


\end{itemize}

The thesis is expected to be finished towards the end  of the spring
semester of 2009.



\end{document}








