\section{Importance sampling}
		\label{sec:importance_sampling}
		In many cases using the Monte Carlo method, we encounter the scenario that we want to compute the expected value of a function $f(\mathbf{X})$, but $f(x)$ yields very small values outside a region $A$, for which the probability to have $\mathbf{X}\in A$ is small. In such a scenario the set $A$ may have a small volume, or may be poorly covered by the distribution $\mathbf{X}$. Therefore a sample from a basic Monte Carlo method from the distribution $\mathbf{X}$ could fail to have even one point inside the region $A$. 

		Clearly we need a better way to get Monte Carlo samples, that is by sampling the important region. To achieve this we sample from a weighted distribution, which prioritizes the important region. This is called importance sampling.  
		
		To implement importance sampling we replace the brute force Metropolis algorithm with a walk in coordinate space biased by the trial wave function, an approach based on the Fokker-Planck equation \cite{fokker1914} \cite{planck1917} and the Langevin equation \cite{langevin1908} for generating a trajectory in coordinate space. The Fokker-Planck equation, named after Adriaan Fokker and Max Planck, is often used as a model for more general Markov processes. 
		%\todo{references}

		\subsection{Importance sampling}

			By using the Fokker-Planck equation and the Langevin equation we can derive and implement importance sampling in our Metropolis algorithm.

			For one particle or walker, a diffusion process characterized by a
			time-dependent probability density $P\left(x,t\right)$ in one dimension,
			we have the Fokker-Planck equation
			\begin{align}
				\frac{\partial P}{\partial t}=D\frac{\partial}{\partial x}\left(\frac{\partial}{\partial x}-F\right)P\left(x,t\right),
			\end{align}
			where $F$ is a drift term and $D$ is the diffusion coefficient.

			The new positions in coordinate space are found using the Langevin
			equation with Euler's method. We go from the Langevin equation
			\begin{align}
				\frac{\partial x(t)}{\partial t}=DF(x(t))+\eta,
			\end{align}
			where $\eta$ is a random variable. This gives us a new position
			\begin{align}
				y=x+DF(x)\Delta t+\xi\sqrt{\Delta t}.
			\end{align}
			Here $\xi$ is a Gaussian random variable and $\Delta t$ is a chosen
			time step. $D$ comes from the factor $1/2$ in the kinetic energy
			operator, and is therefore equal to $1/2$ in atomic units.

			The process of isotropic diffusion characterized by a time-dependent
			probability density $P\left(\mathbf{x},t\right)$ will, as an approximation,
			obey the Fokker-Planck equation
			\begin{align}
				\frac{\partial P}{\partial t}=\sum_{i}D\frac{\partial}{\partial\mathbf{x_{i}}}\left(\frac{\partial}{\partial\mathbf{x_{i}}}-\mathbf{F_{i}}\right)P(\mathbf{x},t),
			\end{align}
			where $\mathbf{F}_{i}$ is component number $i$ of the drift term
			caused by an external potential, and $D$ is the diffusion coefficient.
			We set the left hand side equal to zero and obtain the convergence
			to a stationary probability density
			\begin{align}
				\frac{\partial^{2}P}{\partial{\mathbf{x_{i}}^{2}}}=P\frac{\partial}{\partial{\mathbf{x_{i}}}}\mathbf{F_{i}}+\mathbf{F_{i}}\frac{\partial}{\partial{\mathbf{x_{i}}}}P.
			\end{align}


			Inserting the drift vector, $\mathbf{F}=g(\mathbf{x})\frac{\partial P}{\partial\mathbf{x}}$,
			we get
			\begin{align}
				\frac{\partial^{2}P}{\partial{\mathbf{x_{i}}^{2}}}=P\frac{\partial g}{\partial P}\left(\frac{\partial P}{\partial{\mathbf{x}_{i}}}\right)^{2}+Pg\frac{\partial^{2}P}{\partial{\mathbf{x}_{i}^{2}}}+g\left(\frac{\partial P}{\partial{\mathbf{x}_{i}}}\right)^{2}.
			\end{align}
			To meet the condition of stationary density the left hand side has
			to be zero. This means that the terms containing first and second
			order derivatives has to cancel each other, which is only possible
			if $g=1/P$. This yields
			\begin{align}
				\mathbf{F}=2\frac{1}{\Psi_{T}}\nabla\Psi_{T},
			\end{align}
			known as the quantum force. This so-called force pushes the walker
			towards regions of configuration space where the trial wave function
			is large, thus increasing the efficiency of the simulation. This is
			a great improvement on the Metropolis algorithm where the walker otherwise would have
			the same probability to move in every direction.

			From the Fokker-Planck equation we get a transition probability given
			by Green's function
			\begin{align}
				G(y,x,\Delta t)=\frac{1}{(4\pi D\Delta t)^{3N/2}}\exp\left(-\frac{(y-x-D\Delta tF(x))^{2}}{4D\Delta t}\right).
			\end{align}
			This means that we now have the Metropolis algorithm
			\begin{align}
				A(y,x)=\mathrm{min}(1,q(y,x))),
			\end{align}
			where
			\begin{align}
				q(y,x)=\frac{|\Psi_{T}(y)|^{2}}{|\Psi_{T}(x)|^{2}},
			\end{align}
			is replaced by the Metropolis-Hastings algorithm,
			\begin{align}
				q(y,x)=\frac{G(x,y,\Delta t)|\Psi_{T}(y)|^{2}}{G(y,x,\Delta t)|\Psi_{T}(x)|^{2}}.
			\end{align}
