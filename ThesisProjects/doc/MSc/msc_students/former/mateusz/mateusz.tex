\documentstyle[a4wide]{article}
\newcommand{\OP}[1]{{\bf\widehat{#1}}}

\newcommand{\be}{\begin{equation}}

\newcommand{\ee}{\end{equation}}

\begin{document}

\pagestyle{plain}

\section*{Thesis title: Monte-Carlo simulations of light nuclei}

{\bf The aim of this thesis is to study numerically systems of
light nuclei using various models for the nucleon-nucleon interaction.
Variational Monte Carlo (VMC) and 
Diffusion Monte Carlo (DMC) methods will be used for solving 
Schr\"odinger's equation. }




A microscopic treatment of 
problems in nuclear physics involves a highly complex and state dependent
interaction. For systems of few nucleons (protons and neutrons) ab initio
calculations can be performed. In this thesis the aim is to use 
VMC and DMC approaches. These methods are briefly described below.

\subsection*{Variational Monte Carlo}
The variational quantum Monte Carlo (VMC) has been widely applied 
to studies of quantal systems. 
The recipe consists in choosing 
a trial wave function
$\psi_T({\bf R})$ which we assume to be as realistic as possible. 
The variable ${\bf R}$ stands for the spatial coordinates, in total 
$dN$ if we have $N$ particles present. The variable $d$ is the dimension
of the system. 
The trial wave function serves then as
a mean to define the quantal probability distribution 
\be
   P({\bf R})= \frac{\left|\psi_T({\bf R})\right|^2}{\int \left|\psi_T({\bf R})\right|^2d{\bf R}}.
\ee
This is our new probability distribution function  (PDF). 

The expectation value of the energy $E$
is given by
\be
   \langle E \rangle =
   \frac{\int d{\bf R}\Psi^{\ast}({\bf R})H({\bf R})\Psi({\bf R})}
        {\int d{\bf R}\Psi^{\ast}({\bf R})\Psi({\bf R})},
\ee
where $\Psi$ is the exact eigenfunction. Using our trial
wave function we define a new operator, 
the so-called  
local energy, 
\be
   E_L({\bf R})=\frac{1}{\psi_T({\bf R})}H\psi_T({\bf R}),
   \label{eq:locale1}
\ee
which, together with our trial PDF allows us to rewrite the 
expression for the energy as
\be
  \langle H \rangle =\int P({\bf R})E_L({\bf R}) d{\bf R}.
  \label{eq:vmc1}
\ee
This equation expresses the variational Monte Carlo approach.
For most hamiltonians, $H$ is a sum of kinetic energy, involving 
a second derivative, and a momentum independent potential. 
The contribution from the potential term is hence just the 
numerical value of the potential.
A good discussion of the VMC approach for nuclear systems can be found 
in Ref.~[1].
\subsection*{Diffusion Monte Carlo}
The DMC method is based on rewriting the 
Schr\"odinger equation in imaginary time, by defining
$\tau=it$. The imaginary time Schr\"odinger equation is then
\be
   \frac{\partial \psi}{\partial \tau}=-\OP{H}\psi,
\ee
where we have omitted the dependence on $\tau$ and the spatial variables
in $\psi$.
The wave function $\psi$  is again expanded in eigenstates of the Hamiltonian 
\be
    \psi = \sum_i^{\infty}c_i\phi_i,
    \label{eq:wexpansion_dmc}
\ee 
where 
\be 
   \OP{H}\phi_i=\epsilon_i\phi_i, 
\ee 
$\epsilon_i$ being an eigenstate of $\OP{H}$. 

In order to solve the above equations, 
importance sampling is essential for DMC methods, 
if the simulation is to be efficient. 
A trial or guiding wave function $\psi_T({\bf R})$, which closely
approximates the ground state wave function is introduced.
This is where typically the VMC result enters.
A new distribution is defined as 
\be
   f({\bf R}, \tau)=\psi_T({\bf R})\psi({\bf R}, \tau),
\ee
which is also a solution of the Schr\"odinger equation when  
$\psi({\bf R}, \tau)$ 
is a solution. 
Schr\"odinger's equation is consequently modified to
\be
   \frac{\partial f({\bf R}, \tau)}{\partial \tau}=
    \frac{1}{2}\nabla\left[\nabla -F({\bf R})\right]f({\bf R}, \tau)
    +(E_L({\bf R})-E_T)f({\bf R}, \tau).
    \label{eq:dmcequation2}
\ee
In this equation we have introduced the so-called force-term $F$,
given by
\be
   F({\bf R})=\frac{2\nabla \psi_T({\bf R})}{ \psi_T({\bf R})},
\ee
and is commonly referred to as the ``quantum force''. 
The local energy $E_L$ is defined as previously
\be
    E_L{\bf R})=-\frac{1}{\psi_T({\bf R})}
                \frac{\nabla^2 \psi_T({\bf R})}{2}+V({\bf R})\psi_T({\bf R}),
\ee
and is computed, as in the VMC method, with respect to the trial wave function
while $E_T$ is the trial energy.

We can give the following interpretation to Eq.~(\ref{eq:dmcequation2}).
The right hand side of the importance sampled DMC equation
consists, from left to right, of diffusion, drift and rate terms. The
problematic potential dependent rate term of the non-importance 
sampled method is replaced by a term dependent on the difference 
between the local
energy of the guiding wave function and the trial energy. 
The trial energy is initially chosen to be the VMC energy of 
the trial  wave function, and is
updated as the simulation progresses. Use of an optimised 
trial function minimises the difference between the local 
and trial energies, and hence
minimises fluctuations in the distribution $f$ . 
A wave function optimised using VMC is ideal for this purpose, 
and in practice VMC provides the best
method for obtaining wave functions that accurately 
approximate ground state wave functions locally. 

\subsection*{Nucleon-Nucleon interactions}
Several models for the nucleon-nucleon interaction exists.
All models are essentially based on a meson-exchange picture, see 
e.g., Ref.~[2,3]. The parameters
of these models are chosen so as to reproduce on-shell scattering data.

In Ref.~[1] and in similar studies, a potential model based on an operator
expansion, where the operators depend on the spatial coordinates,
has been employed, see Ref.~[2] for more details.  
In this thesis the plan is to employ another potential model,
the so-called charge-dependent Bonn potential model [3]. This potential
is based on a field-theoretic meson-exchange model and is typically 
modelled in momentum space.

The task is to utilize 
the latter potential model for light nuclei (with mass
$A=3,4$) and compare with those obtained
with the Argonne $V_{18}$ potential.
Moreover, a correlation function based on the two-body scattering matrix
of the Bonn potential will be employed in the calculations.

\section*{Progress plan}
The thesis is expected to be finished towards the end  of the spring 
semester of 2004.
Parts of the thesis project will be done
at Argonne National Lab, USA.
\begin{itemize}
\item Fall 2002: Follows lectures in Fys 211 (3vt), 270 (3vt), 
      Fys 305 (many-body physics, 3vt) and
      computational physics III (3vt) with final exam.  
\item Spring 2003:  Exam in Fys 372 (Nuclear structure, 3vt). Begin
      of thesis work.
      Write a code which solves the variational Monte-Carlo (VMC) problem
      for two interacting nucleons interacting with just a central force.
      This code is then expanded to include the full nucleon-nucleon force 
      and to account for more than two
      nucleons. These nucleons do still interact through a two-body force.
      Construction of a Slater determinant for a system of more than 2
      nucleons.
\item Fall 2003: The next step is to write a Diffusion 
                 Monte Carlo (DMC) programme for the 
      same system. 
      The DMC code receives as input the optimal 
     variational energy and wave function from the VMC calculation and solves
     in principle the Schr\"odinger equation exactly.

\item Spring 2004: Writeup of thesis, thesis exam and exam in Fys 305.

\end{itemize}

\section*{Literature}
\begin{enumerate}
\item J.~A.~Carlson and R.~B.~Wiringa, in Computational Nuclear Physics I,
eds.~K.~Langanke, J.~A.~Maruhn and S.~E.~Koonin, (Springer, Berlin, 1991).
\item R.~B.~Wiringa, V.G.J. Stoks, and R. Schiavilla, Phys. Rev. C 51,
38 (1995).
\item R. Machleidt, Phys. Rev. C 63, 024001 (2001).
\end{enumerate}

\end{document}
