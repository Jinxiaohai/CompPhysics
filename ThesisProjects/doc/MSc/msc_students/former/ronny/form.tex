\documentstyle[aps]{revtex}

\begin{document}

\pagestyle{plain}

\section*{Expressions for the direct URCA process}

Here follows the derivation of the analytical 
expression to be coded  for the direct URCA process.
Throughout I will always set $\hbar = c =1$.
Unless stated, all momenta are four-momenta.
The only approximations which will be used is to assume
that the neutrinos escape (the mean free path of the neutrinos
is typically larger than the radius of the star) and that 
$k_BT$ is musch smaller than the chemical potentials of the 
protons, electrons and neutrons. 
The approximation is necessary in order to establish the standard
equations for $\beta$-stable matter, i.e., 
\begin{equation}
      \mu_n=\mu_e+\mu_p,
\end{equation}
where the labels $n,e,p$ will herafter refer to neutrons, electrons and
protons, respectively. In the labelling below, $\nu$ is the
neutrino. The fact that neutrinos escape will lead to the 
approximation that we neglect the neutrino momentum in the
four-momentum conserving $\delta$-function.  
This simplifies the integral over the angles in the expression
for the emissivity. The second approximations simplifies the integral
over energies with the Fermi-Dirac functions since the term 
$\exp{-\mu_i/k_BT}$ is very small then.  

I will only list the expression for neutron decay,
$n\rightarrow p+e^++\nu_e$, the expression
for electron capture being trivial yielding the same contribution to
the emissivity.

The simplest form for the charged current interaction will be used,
\begin{equation}
      {\cal L}_c = J^{\mu}l_{\mu},
\end{equation}
with the hadronic current as
\begin{equation}
      J_{\mu}=\frac{G}{\sqrt{2}}\overline{u}\gamma_{\mu}(1-g_A\gamma_5)u,
\end{equation}
and the leptonic part as
\begin{equation}
      l_{\mu}=\overline{u}\gamma_{\mu}(1-\gamma_5)u.
\end{equation}
Obviously, the hadronic part is in its simplest form, where terms
like $c_V$ are set equal one. The constant $g_A$ will also need a proper
medium renormalization.

The emissivity of the direct URCA process is given by
\begin{equation}
      \varepsilon_{\nu}=\prod_{i=1}^{4}\int \frac{d^3p_i}{(2\pi)^3}
                  E_{\nu}Wn_n(1-n_e)(1-n_p),
\end{equation}
with 
\begin{equation}
      W=\frac{(2\pi)^4}{\prod_{i=1}^{4}2E_i}\delta^{(4)}(p_n-p_e-p_p-p_{\nu})M^2
\end{equation}
with $M$ being the Feynman matrix and $n_i$ the Fermi-Dirac distribution
functions. 
Inserting the expression for $W$ the emissivity becomes
\begin{equation}
      \varepsilon_{\nu}=\frac{1}{2^{12}\pi^8}\int \frac{d^3p_n}{2E_n}
                  \int \frac{d^3p_p}{2E_p}\int \frac{d^3p_e}{2E_e}
                  \int \frac{d^3p_{\nu}}{2}
                  n_n(1-n_e)(1-n_p)\delta^{(4)}(p_n-p_e-p_p-p_{\nu})M^2.
      \label{eq:emis}
\end{equation}
The squared Feynman matrix is
\begin{equation}
      M^2=\frac{G^2}{2}\left[(p_np_{\nu})(p_ep_p)(1+g_A)^2+
                             (p_np_e)(p_{\nu}p_p)(1-g_A)^2
                             -m_pm_n(p_{\nu}p_e)(1-g_A^2)\right].
      \label{eq:m}
\end{equation}
If you set $g_A=1$ the last two terms vanish and you recover the expression
of Iwamoto, see Eq.\ (3.5) in Ann.\ Phys.\ {\bf 141} (1982) 1.
I will retain them, although that means that we get two additional
integrations to perform.

In the following I will concentrate on the first term, the two others
being similar. In doing the integration over the angles I will follow
Appendix F of Shapiro and Teukolsky.

First, let us look at just the integration over the solid angles $\Omega$.
In Eq.\ (\ref{eq:emis}) this part is given by
\begin{equation}
    \int d\Omega_n \int d\Omega_{\nu} \int d\Omega_e \int d\Omega_p
    \delta^{(3)}({\bf p}_n-{\bf p}_e-{\bf p}_p-{\bf p}_{\nu}).
\end{equation}
We rewrite the $\delta$ function as
\begin{equation}
   \frac{\delta(|{\bf p}_p|-|{\bf p}_n-{\bf p}_e-{\bf p}_{\nu}|)}{|{\bf p}_p|^2}
   \delta(\Omega_p-\Omega_{n-e-\nu}).
\end{equation}
The integral over $\Omega_p$ yields then
\begin{equation}
    \int d\Omega_n \int d\Omega_{\nu} \int d\Omega_e 
    \frac{\delta(|{\bf p}_p|-|{\bf p}_n-{\bf p}_e-{\bf p}_{\nu}|)}{|{\bf p}_p|^2}.
\end{equation}
Choosing
\begin{equation}
      x= {\bf p}_n-{\bf p}_{\nu},
\end{equation} 
and the $z$-axis for $p_e$ along $x$ and 
rewriting the last $\delta$-function as (setting $|{\bf p}_i|= p_i$) 
\begin{equation}
     \delta(|{\bf p}_p|-|{\bf p}_n-{\bf p}_e-{\bf p}_{\nu}|)=
     \frac{\delta(cos\theta_{xe} -(p_p^2-x^2-p_e^2)/(2xp_e))}
     {xp_e/p_p},
\end{equation}
one obtains for the integration over the angles
\begin{equation}
    \frac{2\pi}{|{\bf p}_p||{\bf p}_e|}\int d\Omega_n \int d\Omega_{\nu}
    \frac{1}{|{\bf p}_n-{\bf p}_{\nu}|}.
\end{equation}

The assumption we will make is that $|{\bf p}_n| >> |{\bf p}_{\nu}|$.
One should keep this assumption in mind because if one is to study 
such processes in supernova environment where neutrinos can be rather
energetic, the above approximation is no longer valid.

The above integral is then
\begin{equation}
    \frac{2\pi}{|{\bf p}_p||{\bf p}_n||{\bf p}_e|}\int d\Omega_n \int d\Omega_{\nu}.
    \label{eq:int}
\end{equation}

The integration over the angles depends now on the angle dependence in the 
squared Feynman amplitude. If there is no angle dependence, as assumed by
Shapiro and Teukolsky (and most workers in the field), the integral
is trivial and gives just
\begin{equation}
    \frac{(4\pi)^3}{2|{\bf p}_p||{\bf p}_n||{\bf p}_e|}.
\end{equation}


If we however insist on the angle dependence in Eq.\ (\ref{eq:m}), we need
to do the integration explicitely. Let us just look at the first term of
Eq.\  (\ref{eq:m}) and omit the constants in front of it.
Before we do that, we need to rewrite the first term of Eq.\  (\ref{eq:m}) .
Using the conservation of four momentum
($p_n-p_e-p_p-p_{\nu}=0  $) we have
\begin{equation}
      p_n^2+p_{\nu}^2-2p_{n}p_{\nu}=p_{e}^2+p_{p}^2+2p_{e}p_{p},
\end{equation}
which gives
(using bare masses) 
\begin{equation}
      p_{e}p_{p}=-p_{n}p_{\nu}+\frac{m_n^2-m_p^2-m_e^2}{2}.
\end{equation}
Inserting this in $(p_np_{\nu})(p_ep_p)$ of Eq.\  (\ref{eq:m}) 
we get
\begin{equation}
      (p_np_{\nu})(p_ep_p)=(p_np_{\nu})(-p_{n}p_{\nu}+\frac{m_n^2-m_p^2-m_e^2}{2} ).
      \label{eq:pp}
\end{equation}
Using that
\begin{equation}
      p_np_{\nu}= E_nE_{\nu}-{\bf p}_n{\bf p}_{\nu}=E_nE_{\nu}-p_np_{\nu}cos\theta_{n\nu},
\end{equation}
one sees that Eq.\ (\ref{eq:pp}) is proportional to
\begin{equation} 
     c_0 + c_1cos\theta_{n\nu} + c_2(cos\theta_{n\nu})^2.
\end{equation}
When we do the integration over $\Omega_{\nu}$ in Eq.\ (\ref{eq:int}) we see that
the term with $c_1$ becomes zero and we are left with 
\begin{equation}
    \frac{2\pi}{|{\bf p}_p||{\bf p}_n||{\bf p}_e|}
    \int d\Omega_n \int d\Omega_{\nu}(p_np_{\nu})(p_ep_p)=
    \frac{(4\pi)^3}{2|{\bf p}_p||{\bf p}_n||{\bf p}_e|}(c_0+\frac{1}{3}c_2),
\end{equation}
with 
\begin{equation}
      c_0=\frac{m_n^2-m_p^2-m_e^2}{2}E_nE_{\nu}  - (E_nE_{\nu})^2
\end{equation}
and 
\begin{equation}
      c_2=-(p_np_{\nu})^2=-(E_n^2-m_n^2)E_{\nu}^2
\end{equation}
with the mass of the neutrino set equal zero and 
\begin{equation}
      E_i=\sqrt{p_i^2+m_i^2}.
\end{equation}
Summing up, the integral over the angles gives
\begin{equation}
    \frac{(4\pi)^3}{2|{\bf p}_p||{\bf p}_n||{\bf p}_e|}
    (\frac{m_n^2-m_p^2-m_e^2}{2}E_nE_{\nu}  - (E_nE_{\nu})^2-
     \frac{1}{3}(E_n^2-m_n^2)E_{\nu}^2)).
\end{equation}
Including the constants in front of the squared Feynman amplitude we get
\begin{equation}
    (1+g_A)^2\frac{G^2}{2}\frac{(4\pi)^3}{2|{\bf p}_p||{\bf p}_n||{\bf p}_e|}
    \left(\frac{m_n^2-m_p^2-m_e^2}{2}E_nE_{\nu}  - (E_nE_{\nu})^2-
     \frac{1}{3}((E_n^2-m_n^2)E_{\nu}^2)\right).
     \label{eq:firstterm}
\end{equation}
The second term in Eq.\ (\ref{eq:m}) is evaluated along the same lines and gives
\begin{equation}
    (1-g_A)^2\frac{G^2}{2}\frac{(4\pi)^3}{2|{\bf p}_p||{\bf p}_n||{\bf p}_e|}
    \left(\frac{m_n^2-m_p^2+m_e^2}{2}E_pE_{\nu}  - (E_pE_{\nu})^2-
     \frac{1}{3}((E_p^2-m_p^2)E_{\nu}^2)\right),
     \label{eq:secondterm}
\end{equation}
while the third term is much simpler (since the integration over the angle
yields zero)
\begin{equation}
    -(1-g_A^2)\frac{G^2}{2}\frac{(4\pi)^3}{2|{\bf p}_p||{\bf p}_n||{\bf p}_e|}
    m_pm_nE_eE_{\nu}.
     \label{eq:thirdterm}
\end{equation}

Obviously, if one uses $g_A \approx 1$, the last two terms become
vanishingly small or zero. Effective masses can also be inserted 
instead of the bare masses (note here that the vector part of the self energy
is neglected). I think the question of the renormalization of say the axial
operator in infinite matter is something we will have to defer to later
studies.

The ugly part begins now, namely the integration over the momenta in Eq.\ (\ref{eq:emis}).
Improvements to my brute force approach are heartly welcome.

Let us then look at the first of the above terms.
The integral to be performed is
\begin{eqnarray}
      \varepsilon_{\nu}=\frac{1}{2^{12}\pi^8}\int \frac{p_n^2dp_n}{2E_n}
                  \int \frac{p_p^2dp_p}{2E_p}\int \frac{p_e^2dp_e}{2E_e}
                  \int \frac{p_{\nu}^2dp_{\nu}}{2}
                  n_n(1-n_e)(1-n_p)\delta (E_n-E_e-E_p-E_{\nu})  \cr
      \times (1+g_A)^2\frac{G^2}{2}\frac{(4\pi)^3}{2|{\bf p}_p||{\bf p}_n||{\bf p}_e|}
      \left(\frac{m_n^2-m_p^2-m_e^2}{2}E_nE_{\nu}  - (E_nE_{\nu})^2-
       \frac{1}{3}((E_n^2-m_n^2)E_{\nu}^2)\right).
      \label{eq:emis2}
\end{eqnarray}


Using $E_{\nu}=p_{\nu}$ and for the other particles $E_jdE_j=p_jdp_j$ 
(this also cancels the denominators $|{\bf p}_p||{\bf p}_n||{\bf p}_e|$) one obtains
\begin{eqnarray}
      \varepsilon_{\nu}=\frac{1}{2^7\pi^5}\int dE_n
                  \int dE_p\int dE_e\int E_{\nu}^2dE_{\nu}
                  n_n(1-n_e)(1-n_p)\delta (E_n-E_e-E_p-E_{\nu})\cr
             (1+g_A)^2\frac{G^2}{2}
             \left(\frac{m_n^2-m_p^2-m_e^2}{2}E_nE_{\nu}  - 
             (E_nE_{\nu})^2-\frac{1}{3}((E_n-m_n^2)^2E_{\nu}^2)\right).
      \label{eq:emis3}
\end{eqnarray}
Inserting the Fermi-Dirac functions we have
\begin{eqnarray}
      \varepsilon_{\nu}=(1+g_A)^2\frac{G^2}{2}\frac{1}{2^7\pi^5}\int dE_n
                  \int dE_p\int dE_e
                  \int E_{\nu}^2dE_{\nu}
                  \delta (E_n-E_e-E_p-E_{\nu})\nonumber\cr       
       \times \left(\frac{m_n^2-m_p^2-m_e^2}{2}E_nE_{\nu}  - (E_nE_{\nu})^2-
       \frac{1}{3}((E_n^2-m_n^2)E_{\nu}^2)\right)\nonumber \\
      \times\frac{1}{1+\exp{\beta (E_n-\mu_n)}}
      \times\frac{1}{1+\exp{-\beta (E_e-\mu_e)}}
      \times\frac{1}{1+\exp{-\beta (E_p-\mu_p)}}.
      \label{eq:emis4}
\end{eqnarray}

Now we introduce new variables
$\frac{x_i}{\beta}=E_i-\mu_i$ and $x_{\nu}=E_{\nu}\beta$. We can then rewrite
the latter equation as follows ( defining 
$A(x_n, x_{\nu})=\left(\frac{m_n^2-m_p^2-m_e^2}{2}E_nE_{\nu}  - (E_nE_{\nu})^2-
       \frac{1}{3}((E_n^2-m_n^2)E_{\nu}^2)\right)$)
we obtain
\begin{eqnarray}
      \varepsilon_{\nu}=(1+g_A)^2\frac{G^2}{2}\frac{(k_BT)^5}{2^7\pi^5}
                                 \int dx_n\int dx_p\int dx_e
                                 \int x_{\nu}^2dx_{\nu}
                                 \delta (x_n-x_e-x_p-x_{\nu})A(x_n,x_{\nu})
                                \times\frac{1}{1+e^{x_n}}
                                \frac{1}{1+e^{-x_p}}
                                \frac{1}{1+e^{-x_e}}.
      \label{eq:emis5}
\end{eqnarray}
Performing the integral over $x_e$ we obtain
\begin{eqnarray}
      \varepsilon_{\nu}=(1+g_A)^2\frac{G^2}{2}\frac{(k_BT)^5}{2^7\pi^5}
                                 \int dx_n\int dx_p
                                 \int x_{\nu}^2dx_{\nu}
                                 A(x_n,x_{\nu})
                                \times\frac{1}{1+e^{x_n}}
                                \frac{1}{1+e^{-x_p}}
                                \frac{1}{1+e^{x_n-x_p-x_{\nu}}}.
      \label{eq:emis6}
\end{eqnarray}
The integral over $x_p$ yields then (defining $u=x_n-x_{\nu}$) 
\begin{eqnarray}
      \varepsilon_{\nu}=(1+g_A)^2\frac{G^2}{2}\frac{(k_BT)^5}{2^7\pi^5}
                                 \int dx_n\int x_{\nu}^2dx_{\nu}
                                 A(x_n,x_{\nu})\nonumber\\
                                \times\frac{1}{1+e^{x_n}}\frac{1}{e^{u}-1}
                                \left(ue^u+ln2-e^uln(1+e^u)\right).
      \label{eq:emis7}
\end{eqnarray}
Note well that this expression is more less the same as that for 
the second term in Eq.\ (\ref{eq:secondterm}), the only difference
being that in that case we will perform the integral over $x_n$ and that
the function $A$ now depends on $x_p$ and $x_{\nu}$. The last term 
gives
\begin{eqnarray}
      \varepsilon_{\nu}=m_pm_n(1-g_A^2)\frac{G^2}{2}\frac{(k_BT)^7}{2^7\pi^5}
                                 \int dx_n\int dx_p\int dx_ex_e
                                 \int x_{\nu}^3dx_{\nu}
                                 \delta (x_n-x_e-x_p-x_{\nu})
                                \times\frac{1}{1+e^{x_n}}
                                \frac{1}{1+e^{-x_p}}
                                \frac{1}{1+e^{-x_e}}.
      \label{eq:lastemis1}
\end{eqnarray}
The integrals over $x_n$ and $x_p$ will however be similar to those above.
Most likely these equations can be further simplified with Maple.
The expression given by Prakash {\em et al.} in Phys.\ Rev.\ Lett.\ {\bf 66}
(1991) 2703 is
\begin{equation}
      \varepsilon_{\nu}=\frac{457\pi}{10080}G_F^2cos^2\Theta_C(1+3g_A^2)
      \frac{m_nm_p\mu_e}{\hbar^{10}c^5}(k_BT)^6\theta(p_F^e+p_F^p-p_F^n).
\end{equation}


\end{document}

