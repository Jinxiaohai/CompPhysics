\documentstyle[a4wide]{article}
\newcommand{\OP}[1]{{\bf\widehat{#1}}}

\newcommand{\be}{\begin{equation}}

\newcommand{\ee}{\end{equation}}

\begin{document}

\pagestyle{plain}

\section*{Thesis title: Coupled-cluster and density functional theory studies of quantum dots}

The aim of this thesis is to study numerically systems consisting of several
interacting electrons in two dimensions, confined to small regions
between layers of semiconductors. 
These electron systems
are dubbed quantum dots in the literature. 
Semiconductor quantum dots are structures where
charge carriers are confined in all three spatial dimensions, 
the dot size being of the order of the Fermi wavelength 
in the host material, typically between  10 nm and  1 $\mu$m.
The confinement is usually achieved by electrical gating of a 
two-dimensional electron gas (2DEG), 
possibly combined with etching techniques. Precise control of the
number of electrons in the conduction band of a quantum dot 
(starting from zero) has been achieved in GaAs heterostructures. 
The electronic spectrum of typical quantum dots
can vary strongly when an external magnetic field is applied, 
since the magnetic length corresponding to typical 
laboratory fields  is comparable to typical dot sizes.
In coupled quantum dots Coulomb blockade effects, 
tunneling between neighboring dots, and magnetization 
have been observed as well as the formation of a
delocalized single-particle state. 

Quantum dots have been used to fabricate  quantum gates and are also used in the emerging field
of quantum nano medicine.  

More specifically, this thesis aims at studying the reliability 
of the coupled-cluster method for studies of quantum
dots in two dimensions. The coupled-cluster method has been extremely succesful in providing almost
exact ab initio results in quantum chemistry, atomic physics, molecular physics and nuclear physics.
These calculations will in turn provide the basis for determining a as good as possible
ground state wave function. This wave function will in turn be used to define the quantum
mechanical density.  The density will be used to construct a density functional for quantum dots
using the adiabatic-connection method as described by Teale {\em et al} in J.~Chem.~Phys.~{\bf 130},
104111 (2009).  The results will be compared with existing density functional for quantum dots.

The reliability of the coupled-cluster method as function of the externally applied magnetic field
will be compared with ab initio Monte Carlo and large-scale diagonalization techniques. The latter
two calculations will be performed by other Master of science students.

The results are expected to be published in leading journals.

The aims of this thesis are as follows

\begin{itemize}
\item Develop first a Hartree-Fock code for electrons trapped in a single harmonic oscillator  
in two dimensions.   This part entails developing a code for computing the Coulomb interaction
in two dimensions in the laboratory system.
\item Compare the Hartree-Fock results with available large scale diagonalization techniques.
\item The Hartree-Fock interaction is then used as input to our exsisting coupled-cluster codes.
The results will be compared with large scale diagonalization and Monte Carlo 
techniques for 2, 6, 12 and 20 
electrons in a single harmonic oscillator well.
 \item The obtained ground states will in turn be used to define a as exact as possible 
density functional for quantum dots using the adiabatic-connection method. The density functional can in turn be used to model
systems with a large number of quantum dots. Possible applications of these quantum dots functionals are studies of applications to solar cells.
\end{itemize}
 

\section*{Progress plan and milestones}
The thesis is expected to be handed in June 1 2010.
\begin{itemize}
\item Fall 2009: Develop a Hartree-Fock code for quantum dots and derive a
self-consistent interaction to be used in the coupled cluster codes.
Make comparisons with large-scale diagonalization and Monte Carlo techniques.
\item Spring 2010: Derive a density functional for quantum dots and compare with existing parameterizations. Writeup of thesis and final thesis exam.

\end{itemize}

\begin{thebibliography}{999}

\bibitem {ref1}  A.~M.~Teale, S.~Coriani, and T.~Helgaker, J.~Chem.~Phys.~{\bf 130},
104111 (2009).


\end{thebibliography}



\end{document}


