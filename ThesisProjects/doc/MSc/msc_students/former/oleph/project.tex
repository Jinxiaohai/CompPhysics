\documentstyle[a4wide]{article}
\newcommand{\OP}[1]{{\bf\widehat{#1}}}

\newcommand{\be}{\begin{equation}}

\newcommand{\ee}{\end{equation}}
\newcommand{\bra}[1]{\left\langle #1 \right|}
\newcommand{\ket}[1]{\left| #1 \right\rangle}
\newcommand{\braket}[2]{\left\langle #1 \right| #2 \right\rangle}


\begin{document}

\pagestyle{plain}

\section*{Computational Physics thesis: Structure calculations of Carbon isotopes}

In the past years, $^{16}$C has been 
at the center of two concerns: the shell-model description of $\pi 0s1p-\nu 1s0d$
($\pi$ for protons and $\nu$ for neutrons)
cross-shell nuclei and its unexpectedly low $0^+\rightarrow 2^+$ E2 excitation strength.
In a recent experimental proposal at the National Superconducting Cyclotron laboratory at Michigan
State University (proposal 05107 at PAC29  in december 2005), it was 
proposed to study the structure $^{16}$C ground state 
through the one-proton knockout reaction $^9$Be($^{16}$C,$^{15}$B)X. The experiment has been performed 
recently and the analysis of the data is now being performed in order to 
determine the occupancy of the $p_{3/2}$ orbital, 
expected to be strongly reduced with respect to shell-model
predictions. This experiment will also contribute to 
the spectroscopy of $^{15}$B. Observed gamma rays in $^{15}$B, possibly
from states beyond the $sp$ shell-model space, may provide direct evidence for $sd$ 
shell admixture to the
$^{16}$C ground state. 
In addition to this reaction, the experiment measured  the $0^+\rightarrow 2^+$ 
inelastic excitation of $^{16}$C
in order to determine the nuclear and Coulomb deformations of $^{16}$C through a novel
approach.
The experimenters expect to publish their results not later than a year from now.
The aim of this master thesis is therefore to perform large-scale shell-model calculations
for model spaces containing both the $0p$ and the $1s0d$ shells in order to test 
the importance of these  additional degrees of freedom. Calculations will be mounted for
ground states and low-lying excited states of both $^{16}$C and $^{15}$B
in order to compare the theoretical predicitions with the recent experimental data.

The aims of the thesis are     

\begin{itemize} 
\item Derive new effective interactions for this mass region spanning over two
major shells, the $0p$ and the $1s0d$ shells, with both two-body and three-body forces.
\item Perform large-scale shell-model calculations for these nuclei using the Oslo
shell-model code.  Of particular interest here is to study the effect of spurious center-of-mass
correlations using a newly developed technology by Torgeir Engeland 
\item If the results yield interesting additions to experiment, we expect that the
calculations can be published in  Physical Review C 
\end{itemize}



The activity will be based within the 
Computational Quantum Mechanics project at the Department of Physics and 
CMA. The main location is the Nuclear Physics group. 
The Computational Quantum Mechanics group consists  
presently of two full professors (Hjorth-Jensen and Osnes), one professor Emeritus (Engeland), 
one Adjunct Professor (Dean, Oak Ridge National Lab and CMA), one post-doc (Gaute Hagen, CMA) five PhD
students (CMA(3), Dept ot Physics (2)) and six Master students (Dept of Physics).  

\section*{Progress plan and milestones}
The thesis is expected to be finished towards the end  of the spring 
semester of 2008
\begin{itemize}
\item Fall 2007: Develop new effective interactions for the $0p$ and $1s0d$ shells
using both two and three-nucleon forces.
\item Spring 2003:  Perform shell-model calculations with and without center-of-mass
corrections. Study E2 transitions.
Writeup of thesis and final thesis exam
\end{itemize}

\end{document}





































