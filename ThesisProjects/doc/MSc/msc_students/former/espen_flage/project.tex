\documentstyle[a4wide]{article}
\newcommand{\OP}[1]{{\bf\widehat{#1}}}

\newcommand{\be}{\begin{equation}}

\newcommand{\ee}{\end{equation}}
\newcommand{\bra}[1]{\left\langle #1 \right|}
\newcommand{\ket}[1]{\left| #1 \right\rangle}
\newcommand{\braket}[2]{\left\langle #1 \right| #2 \right\rangle}


\begin{document}

\pagestyle{plain}

\section*{Many-electron correlations effects in quantum dots}

In recent years, low-dimensional semiconductor structures containing a small number of electrons
in the size-quantized conduction band have attracted much attention. 
The main reason for this increased interest is the progress made in novel technologies for
fabricating nanostructures with a very high degree of precision. 
Semiconductor quantum dots are structures where
charge carriers are confined in all three spatial dimensions, 
the dot size being of the order of the Fermi wavelength 
in the host material, typically between  10 nm and  1 $\mu$m.
The confinement is usually achieved by electrical gating of a 
two-dimensional electron gas, 
possibly combined with etching techniques. Precise control of the
number of electrons in the conduction band of a quantum dot 
(starting from zero) has been achieved in GaAs heterostructures. 
The electronic spectrum of typical quantum dots
can vary strongly when an external magnetic field is applied, 
since the magnetic length corresponding to typical 
laboratory fields  is comparable to typical dot sizes.


Another topic which also has received quite a  lot of interest recently, essentially due to the 
possibilities that quantum dots offer as building blocks for quantum information processing, are so-called
coupled quantum dots. 
In coupled quantum dots Coulomb blockade effects, 
tunneling between neighboring dots, and magnetization 
have been observed as well as the formation of a
delocalized single-particle state. 
Coupled quantum dots  provide a powerful source of 
deterministic entanglement between qubits of localized 
but also of delocalized electrons. With such quantum gates it is
possible to create a singlet state out of two electrons 
and subsequently separate (by electronic transport) 
the two electrons spatially with the spins of the two electrons still being
entangled-the prototype of an EPR pair. 
This has opened  up the possibility to study a new class 
of quantum phenomena in electronic nanostructures 
such as the entanglement and
non-locality of electronic EPR pairs, tests of Bell inequalities and quantum cryptography.
Simple quantum circuits, the possible building blocks
for quantum computing based on quantum dots have recently been constructed, see for example
the article of Taylor {\em et al}, Nature Physics {\bf 1}, page 177.


 From a theoretical point
of view, quantum dot systems are a valuable source of novel
quantum effects. Many of these effects are due to the  fact that the
electron-electron interaction and external magnetic field yield enhanced effects compared 
to atoms and molecules.  This raises new
challenges for the theoretical methods, and the validity of
approximations such as popular mean-field approaches can be questioned. For this reason,
quantum dot systems serve as perfect test cases to develop electronic structure
methods, with the results applicable to a great variety of physical
problems. In addition, many
of the system parameters are tunable, for example, the electron number in a quantum dot
can be changed one by one and the confinement potential can be varied
by external gates.


This project focuses on the development appropriate many-body methods for studying the
structure and dynamics of quantum dot system. The methods that will be emphasized are large-scale
diagonalization techniques and Diffusion and Green's function Monte Carlo techniques. 
The topics here can form the basis for at least two PhD theses. These projects enter also an ongoing
research program on computational quantum mechanics in close collaboration with the Center of Mathematics
for Applications (CMA). Two PhD students at CMA (Simen Kvaal and Torquil McDonald S\o rensen) are working
on similar and related topics.

The large-scale diagonalization technique 
(configuration interaction methods in atomic physics and shell-model methods
in nuclear physics) has been utilized and developed by the nuclear physics group at UiO. 
This part of the project has therefore 
strong links with the ongoing ctivity in nuclear structure as well, since quantum dots and nuclei share
several many-body aspects. The topics we wish to explore are listed below.
following topics:
\begin{itemize} 
\item The study of effective interactions  and model spaces for systems of many electrons in quantum dots.
These techniques have been developed over many years within the nuclear physics community, but have few or none
applications in solid state physics. Since electrons in quantum dots are, from a theoretical modeling point of
view, very similar to selfbound nuclear systems, we believe that such techniques can be 
applied to quantum dots as well
and allow for ab initio studies of many-electron dots.
\item The effect of spurious center of mass motion is important for both nuclear systems and confined electronic
systems. Both systems share several similarities 
since nucleons in a nucleus and electrons in a quantum dot can be approximated by harmonic
oscillator wave functions. How to deal with the spurious center of mass motion is an usettled topic in 
nuclear structure studies. 
\item Quantum dots are normally described via a two-dimensional confining harmonic oscillator. This is an 
approximation to the experimental situation, since in principle one should treat the system as a deformed
three-dimensional harmonic oscillator. Our methodology allows also for the exploration of these
degrees of freedom as well.
\item Based on the large-scale diagonalization techniques, one can study the time-development of the wave functions
of different states under the influence of external and time-dependent fields. This is of great interest in order to understand
how such systems evolve in time.
\item  Since the dimensionalities will be huge (typical sizes are $10^9$ basic states or more), one needs to develop 
efficient parallelization approaches  of our present shell-model codes for nuclear systems and electronic systems.
\end{itemize}

The large-scale diagonalization techniques, if converged as function of the size of the model space, yield in principle
an exact solution for a many-body system, be it either a nuclear system or a confined electronic system such as a quantum dot.
It serves therefore as a benchmark  for other methods and tests of the validity of mean-field approaches.
However, large-scale diagonalization techniques reach quickly a limit for how large systems one  can address.
Monte Carlo methods offer then a possibility to extend the number of degrees of freedom and therey study systems composed
of many more particles.
The other set of many-body methods we need to develop are thus Diffusion and Green's function Monte Carlo methods.
These methods allow also in principle an exact solution to the many-particle Schr\"odinger equation.
We have already developed a large Diffusion and Variational Monte Carlo code for studies of bosons,  with an
emphasis on  studies of dilute atoms and Bose-Einstein condensation.
Within this of methods we wish in particular to apply these codes to 
\begin{itemize} 
\item the development  of a fermionic Diffusion Monte Carlo code for electronic systems, quantum dots in particular.
Aided by the results from the large-scale diagonalization part one can use those results for 
constructing good trial wave functions for the Monte Carlo part.
\item  For this project as well, one needs to develop 
efficient parallelization approaches. This part of the project will have a large overlap with an exisiting 
project in computational science (Automatic parallelization of serial codes).
\end{itemize}


The projects which are described here will allow one to make precise statements about quantal many-body systems of great
current interest. The studies of quantum dots is also of high relevance for advances in nanotechnologies.
The projects overlap also with several disciplines and ongoing projects in computational science and numerical mathematics.

The activity will be based within the Computational Quantum Mechanics project at the Department of Physics and 
CMA. The main location is the Nuclear Physics group. 
The Computational Quantum Mechanics group consists  
presently of two full professors (Hjorth-Jensen and Osnes), one professor Emeritus (Engeland), 
one Adjunct Professor (Dean, Oak Ridge National Lab and CMA), one post-doc (M\o ll Nilsen, CMA) seven PhD
students (CMA(3), Dept ot Physics (2), USIT (1) and UiB (1)) and five Master students (Dept of Physics).  

The local supervisors are Susanne Viefers and Morten Hjorth-Jensen.
Moreover, if the approved, we plan to include Prof.~Yoram Alhassid of Yale University as a co-advisor.
Yoram Alhassid has a long-standing research experience within quantum dots. In this connection, a year at
Yale is planned as part of the PhD.

\end{document}
