\chapter{Introduction}
This is not the first time quantum dots have been studied using the coupled-cluster machinery.
In fact, some of the calculations done here have also been done previously.
Quantum dots, that is, strongly confined electrons show a variety of interesting properties.
Of relevance in both experiments and various technical components, is the possibility to fine tune their electrical and optical properties.
 
{\bf trengs bedre overgang}
A coupled-cluster approach aims to find the ground-state energy by distributing the particles in a number of basis states.
In practice working in a truncated basis, coupled cluster also defines a truncation in the possible ways to distribute the particles.

Earlier master's projects have solved the problem, through serial C++ implementations~\cite{marte,mplohne,ymwang}, for up to 20 electrons within 420 basis states.
Programs met convergence problems for increasing number of particles and reduced potential strengths.
An extension was thus natural in the direction of exploring a parallel approach to increase system sizes and at the same time try to overcome the convergence issues.

\paragraph{}
The first part of this thesis, chapter~\ref{ch:qm} and~\ref{ch:manybody}, serves as a theoretical introduction to the basic theory of quantum mechanics and many-body theory.
Important features and terminology are discussed, with a focus on theoretical topics that are required for the understanding of later parts.

Continuing we discuss different systems, quantum dots in particular, in chapter~\ref{ch:qDots}.
Various examples on how systems are implemented by sub-classing the `System' base class are given, and we shed some light on how a system can be optimized, both in  terms of memory and processing requirements, without compromising the flexibility and generalness of the code.

In chapter~\ref{ch:CC} the coupled-cluster method is introduced.
Beginning with a more shallow outline of the method, we will eventually derive the full set of non-linear equations.
In addition to explaining how to implement coupled cluster, the Hartree-Fock method is also briefly mentioned.

The last theoretical chapter is chapter~\ref{ch:OpenCL}, leaning more in a programming technical direction.
OpenCL, a standard and a library for accelerated code, is introduced prior to its application, namely accelerated matrix multiplication.
{\bf tenk p\aa\ bedre overgang her ogs\aa\ }
Matrix multiplication is decoupled from the main program, put in its own class, to make it easier to employ different algorithms without changing the entire code base.

\paragraph{}
Results along with a discussion and a final conclusion close this thesis.
We present results for up to 56 particles in more than 900 basis functions, and for previously unobtainable weak strengths of the confining single-particle potential.
Results and time usage are benchmarked, both against other coupled-cluster programs and other methods.
Final remarks are made for potential extensions to this project.

%Unnecessary
%Some parts were planned for inclusion here, but were omitted due to the limitations a one-year master's project has, whereas other parts clearly stand out as an independent thesis.