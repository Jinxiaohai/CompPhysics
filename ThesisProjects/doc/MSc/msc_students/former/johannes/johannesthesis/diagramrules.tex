\chapter{Diagram rules} \label{diagramregler}

\begin{enumerate}
\item There are $n+1$ vertices, one vertice for each time, with the ordring
$t<t_1<t_2 < \cdots < t_n.$ Each vertex/interaction is represented by a 
dashed line, as in Fig.(\ref{forsteordendiagram}). 

\item Lines with upward pointing arrows are particles and lines with 
downward pointing arrows are holes. Lines starting and ending at the same 
vertex ate holes.

\item Each vertex gives a factor $\frac{1}{2}V_{\alpha\beta\gamma\delta}.$

\item There is an overall sign $(-1)^{n_h+n_l},$ where $n_h$ is the number
of hole lines and $n_l$ is the number of fermion loops.

\item For each interval between two successive vertices there are an energy
factor 

\beq
\left[\sum_h \epsilon_h - \sum_p \epsilon_p \right]^{-1}
\eeq

Where the sum over $h$ is over all hole lines in the intervall and the sum 
over $p$ is over all particle lines in the intervall.

\item For each pair of lines that begins at the same interaction line and 
end at the same interaction line gives a factor $\frac{1}{2}.$

\item All the above factors have to be multiplied. Sum over all the labels 
of fermion lines. 











\end{enumerate}
