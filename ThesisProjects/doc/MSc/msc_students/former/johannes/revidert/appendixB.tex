\chapter{Plane waves and spherical waves}\label{partiellbolge}

When transforming the potential to momentum basis, it is very useful to use an
expansion for the product \sd \braket{\bold x}{\bold k}\sd.  In order to get
this transformation we will have to look at both plane waves, spherical waves
and the connection between these two.  In a free particle state the Hamiltonian
consists just of the kinetic energy operator, and obviously also commutes with
the momentum operator, with the eigenvalue $ \bold k$. The free
particle Hamiltonian commutes also with the operators \sd \bold L^2\sd\, and
\sd L_z\sd, we can then find an eigenket of \sd H_0, \bold L^2\sd\, and \sd
L_z\sd\, denoted \sd \ket{Elm}\sd, here the spin is suppressed. This state is
called a spherical wave state. As a free state can be regarded as a
superposition of various plane wave states \sd \ket{\bold k}\sd\,  with
different \sd\bold k\sd, the same can be done with spherical wave states, but
here with various \sd E,l\sd\, and \sd m\sd. A free particle state can be analyzed by plane wave states or
spherical wave states.\\ 
\\
There should be a connection between a plane wave basis and a spherical wave basis, this connection which may transform a plane wave basis to a spherical wave basis will be derived. Since the complete spherical wave basis is orthonormal each state satisfies the condition 
\beq
\braket{E'l'm'}{Elm}=\delta_{ll'}\delta_{mm'}\delta(E'-E).
\eeq
Since we have an complete basis we can expand a plane wave state in a spherical wave basis as
\beq
\ket{\bold k}= \sum_{lm}\int dE \braket{Elm}{\bold k}\ket{Elm}.
\eeq
We need to find the transformation coefficient $\braket{Elm}{\bold k}$. It is 
helpful to first consider a plane wave state whose propagation is along the z 
axis, \sd \ket{k\bold{ \hat z}}\sd. A crucial property of this state is that 
there is no orbital momentum in the z direction
\beq
L_z\ket{k\bold{ \hat z}}=0.
\eeq
The expansion of \sd \ket{k\bold{ \hat z}} \sd \, is 
\be
\ket{k\bold{ \hat z}}=\sum_{l}\int dE \ket{E,l,m=0}\braket{E,l,m=0}{k\bold{ \hat z}}
\label{eq:kz}
\ee
A general eigenket \sd\ket{\bold k}\sd\, can be obtained by applying a rotation operator on Eq. \eqref{eq:kz},
\beq
\ket{\bold k}= \mathcal R\ket{k\bold{\hat z}}.
\eeq\\
\\
Let us now multiply \sd \bra{Elm}\sd\, on \sd \ket{\bold k}\sd.
\beq
\begin{split}
&\braket{Elm}{\bold k}= \sum_{l'}\int dE'\bra{Elm} \mathcal R\ket{E',l',m'=0}\braket{E',l',m'=0}{k\bold{ \hat z}}\\
&=\sum_{l'}\int dE' \mathcal{R}^l_{m0}\delta_{ll'}\delta(E-E')\braket{E',l',m'=0}{k\bold{\hat z}}\\
& = \mathcal{R}_{m0}^l\braket{Elm=0}{k\bold{\hat z}}. 
\end{split}
\eeq
In order to solve this we observe that the coefficient \sd
\braket{Elm=0}{k\bold{\hat z}}\sd\, is independent of the angles \sd
\theta\sd\, and \sd \phi\sd. We can then postulate that it is on the form \sd
\sqrt{2l+1/4\pi}g_{lE}*(k)\sd. Since the spherical harmonics \sd
Y^m_l\sd\,  are defined as \sd \sqrt{2l+1/4\pi}\mathcal{R}^l_{m0}\sd.
We can write the transformation coefficient \sd \braket{\bold k}{Elm}\sd\, as
\beq
\braket{\bold k}{Elm}=g_{lE}(k)Y^m_l(\bold{\hat k}).
\eeq 
The function \sd g_{lE}(k)\sd\, is the last part to determine. This is done by 
observing that 
\beq
(H_0-E)\ket{Elm}=0,
\eeq
and by doing the same on the eigenbra \sd \bra{\bold k}\sd
\beq
\bra{\bold k}(H_0-E)=\bra{\bold k}\left(\frac{ k^2}{2m}-E\right).
\eeq
Multplying \sd\ket{Elm}\sd\, from right should give zero,
\beq
\left(\frac{ k^2}{2m}-E\right)\braket{\bold k}{Elm}=0.
\eeq
It follows that \sd\braket{\bold k}{Elm}\sd\, is only nonvanishing when
\beq 
E=\frac{ k^2}{2m}.
\eeq
We can then write 
\beq
g_{lE}(k)=N\delta(\frac{k^2}{2m}-E),
\eeq
where \sd N\sd\, is a normalization constant which can be found by considering the orthonormalization condition for \sd\braket{E'l'm'}{Elm}\sd. It turns out that 
\beq
N=\frac{1}{\sqrt{mk}}. 
\eeq
And hence 
\be
\braket{\bold k}{Elm}=\frac{1}{\sqrt{mk}}\delta(\frac{k^2}{2m}-E)Y^m_l(\bold{\hat k}).
\ee
In order to get the transformation in coordinate space we have to use the fact that the wave function for a free spherical wave is \sd j_l(kr)Y^m_k(\bold{\hat r})\sd, where \sd j_l(kr)\sd\, is the spherical Bessel function of order l.\\
\\
The transformation coefficient, $\braket{\bold x}{Elm}$ is then on the form
\be
\braket{\bold x}{Elm}=c_lj_l(kr)Y^m_l(\bold{\hat r}),
\label{eq:xsphere}
\ee
where \sd c_l\sd\, has to be determined. It is determined by comparing Eq. \eqref{eq:xsphere} with \sd \braket{\bold x}{\bold k}\sd. We find that \sd c_l= i^l\sqrt{2mk/\pi}\sd.
