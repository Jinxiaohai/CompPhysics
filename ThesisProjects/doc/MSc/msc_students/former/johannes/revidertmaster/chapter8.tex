\chapter{The two-body matrix elements}

Until now we have mostly been concerned with different theories applied in the
project, in this part we will present how  the calculations are done.  In the
first part we will show explicitly how matrix elements are calculated while we
in the next sections will go through the calculations of the interactions.

\section{Calculation of matrix elements} When computing matrix elements the
Wigner-Eckart theorem has turned out to be very important. The theorem states
that when calculating the matrix element of a spherical tensor  it is allowed to
do a separation in the part that only depends on the projection quantum numbers
and another part that depends on the radial properties.
\beq
\bra{\alpha j m}T^{(k)}_\kappa \ket{\beta j' m'}=(-1)^{j-m}\begin{Bmatrix}j & k & j'\\-m & \kappa & m'\end{Bmatrix}\bra{\alpha j}T^{(k)}\ket{\beta j'},
\eeq
where the $j$'s indicate angular momentum and the $m$'s are the projections, 
while $k$ denotes the rank of the tensor $T$ and $\kappa$ its projection. The 
curly bracket is a 6 $j$-symbol.\\
This theorem can be used to calculate two body matrix elements, we can consider a tensor operator, \sd T=T(1)\otimes T(2) \sd, acting on two independent subsystems denoted 1 and 2. 
\be
\bra{\alpha_1 j_1 \alpha_2 j_2JM}T\ket{\beta_1 j_1'\beta_2j_2'JM}.
\label{eq:wigner}
\ee
By uncouple the wave functions applying Wigner Eckart theorem we rewrite the matrix element in Eq.~\eqref{eq:wigner} as
%\beq
%\begin{split}
%&\bra{\alpha_1 j_1 \alpha_2 j_2JM}T\ket{\beta_1 j_1'\beta_2j_2'J'M'}=(-1)^{J-M}\begin{pmatrix}J & m & J'\\ -M &m'&M'\end{pmatrix}\\
%&\times \bra{\alpha_1 j_1 \alpha_2 j_2J}T\ket{\beta_1 j_1'\beta_2j_2'J'}.
%\end{split}
%\eeq
\beq
\begin{split}
&\bra{\alpha_1 j_1 \alpha_2 j_2JM}T\ket{\beta_1 j_1'\beta_2j_2'JM}\\
&=\sum_{m_1,m_2,\kappa_1,\kappa_2} \braket{j_1m_1,j_2m_2}{JM}\braket{j_1'm_1',j_2'm_2}{J'M'}(-1)^{j_1-m_1}\begin{Bmatrix}j_1&k_1&j_1'\\-m_1&\kappa_1&m_1'\end{Bmatrix}\\
&\times(-1)^{j_2-m_2}\begin{Bmatrix}j_2 & k_2 & j_2'\\ -m_2&\kappa_2 &m_2'\end{Bmatrix}\\
&\times \bra{\alpha_1j_1m_1}T^{k_1}_{\kappa_1}\ket{\beta_1j_1'm_1'}\bra{\alpha_2j_2m_2}T^{k_2}_{\kappa_2}\ket{\beta_2j_2'm_2'}
\end{split}
\eeq
We use Wigner-Eckart theorem again for \sd j_1,\, j_2,\,m_1\sd\, and \sd m_2\sd\, to obtain
\beq
\bra{\alpha_1 j_1 \alpha_2 j_2JM}T\ket{\beta_1 j_1'\beta_2j_2'JM}=\hat J \hat J' \hat k\begin{Bmatrix}j_1&j_2&J\\j_1'&j_2'&J'\\k_1&k_2&k\end{Bmatrix}
\bra{\alpha_1j_1}T_1\ket{\beta_1j_1'}\bra{\alpha_2j_2}T(2)\ket{\beta_2j_2'},
\eeq
where the symbols with a hat are defined as $\hat I=\sqrt{2I+1}$.
The intermediate steps are omitted. For a complete derivation see Ref. \cite{heydeshellmodel}.

\section{Calculating the interactions}
\label{sec:calc_int}



 When computing the two-body matrix element 
\beq
\bra{k_1k_2}v_{12}\ket{k_1'k_2'}
\eeq 
It is convenient to do calculations in relative and center of mass coordinates, where we define the relative momenta
\be
\bold k=\frac{1}{2}|\bold k_1-\bold k_2|
\label{eq:relk}
\ee
and the center-of-mass momenta
\be
\bold K=(\bold k_1+\bold k_2).
\label{eq:centerk}
\ee
 Since the potential is just a function of relative 
coordinates and diagonal in center of mass coordinates the interactions will be on the form
\beq
\bra{kK}v(k,k')\ket{k'K'}=\bra{k}v(k,k')\ket{k'}\delta_{K,K'}.
\eeq
We need a method to transform the interactions in relative and center-of-mass-momenta to laboratory coordinates.
In quantum mechanics a general transformation is done by expanding our initially ket $a$ in an orthonormal basis $\alpha$.
\beq
\ket{a}= \sum_\alpha \ket{\alpha}\braket{\alpha}{a}= \sum_\alpha \braket{\alpha}{a}\ket{\alpha}.
\eeq
In order to find the correct transformation we need to find the transformation coefficients \sd \braket{\alpha}{a} \sd.
When computing the matrix element \sd \bra{a}v\ket{b}\sd, we expand both the ket side and the bra side in the same orthonormal basis, 
\beq
\bra{a}v\ket{b}=\sum_{\alpha,\beta} \braket{a}{\alpha}\bra{\alpha}v\ket{\beta}\braket{\beta}{b}.
\eeq
We are now ready to make the transformations.
\be
\begin{split}
&\ket{k_a l_a j_a k_b l_b j_b T_z J} = \sum_{l,L,j \mathcal J} \int d^3k \int d^3K \begin{Bmatrix}l_a & l_b & \lambda\\\frac{1}{2} & \frac{1}{2} & S\\ j_a & j_b & J \end{Bmatrix}\\
&\times (-1)^{\lambda+\mathcal J-L-S}F\hat{\mathcal J} \hat \lambda^2\hat j_a \hat j_b \hat S\begin{Bmatrix}L & l & \lambda \\ S & J & \mathcal J\end{Bmatrix}\\
& \times\braket{k l j K L \mathcal J, T_z J}{k_a l_a j_a k_b l_b j_b T_z J}\ket{k l j K L \mathcal J T_z J}.
\end{split}
\label{eq:brackettransform}
\ee
The term \sd \braket{k l j K L \mathcal J, T_z J}{k_a l_a j_a k_b l_b j_b T_z
K}\sd\, is the transformation coefficient from the relative system to the
laboratory system. The factor \sd F\sd\, equals 1 for different particles
$(T_z=0)$ and equals \sd (1-(-1)^{l+S+T_z})/\sqrt{2}\sd\, for identical
particles. Here we consider only protons and neutrons, if the $\Delta$ particle with isospin $3/2$ is included it is not always the case that the isospin sums to zero for different particles.\\
The problem is to find the transformation
coefficients.\\
From Eqs \eqref{eq:relk} and \eqref{eq:centerk} we obtain the relations
\begin{equation*}
		\begin{split}
				&\bold k_1=-\bold k +\frac{K}{2}=\bold \rho_1(\bold k,\bold K)\\
				& \bold k_2=\bold k + \frac{K}{2}=\bold \rho_2(\bold k,\bold K).
		\end{split}
\end{equation*}
The two particle bra state $\bra{\bold k_1 \bold k_2}$ is expanded in a partial wave basis as
\begin{equation*}
		\bra{\bold k_1 \bold k_2}=\frac{1}{k_1k_2}\sum_{l_1l_2\lambda \mu}\bra{k_1l_1k_2l_2,\lambda \mu}\{Y^{l_1}(\hat k_1)\times Y^{l_2}(\hat k_2) \}.
\end{equation*}
The state $\bra{\bold k \bold K}$ is similar, we take the scalar product of $\bra{\bold k \bold K}$ and $\ket{\bold k_1 \bold k_2}$ and get
\begin{equation}
		\begin{split}
				&\braket{\bold k \bold K}{\bold k_1 \bold k_2}=\frac{1}{kKk_1k_2}\sum_{\lambda \mu \lambda' \mu'}\sum_{lLl_1l_2}\{Y^l(\hat k)\times Y^L(\hat K)\}^\lambda_\mu\\
				&\{Y^{l_1}(\hat k_1)\times Y^{l_2}(\hat k_2) \}^{\lambda'*}_{\mu'}\braket{klKL,\lambda}{k_1l_1k_2l_2.\lambda}.
		\end{split}
		\label{eq:multitrans}
\end{equation}
By looking at the left side in the above equation we see that it obeys the two particle state orthogonality relations 
\begin{equation*}
		\braket{\bold k \bold K}{\bold{k_1 k_2}}=\delta(\bold k_1-\rho_1(\bold k,\bold K))\delta(\bold k_2-\rho_2(\bold k,\bold K)).
\end{equation*}
The explicit expression for the vector bracket $\braket{\bold k l \bold K L,\lambda}{\bold k_1 l_1 \bold k_2 l_2, \lambda}$ can be obtained by multiplying each side of Eq \eqref{eq:multitrans} with
\begin{equation*}
		\sum_{l_1l_2\lambda'\mu'}\{Y^{l_1'}(\hat k_1)\times Y^{l_2'}(\hat k_2) \}^{\lambda'*}_{\mu'}\{Y^{l_1'}(\hat\rho_1(\bold k,\bold K))\times Y^{l_2'}(\hat \rho_2(\bold k,\bold K) \}^{\lambda'}_{\mu'}
\end{equation*}
and integrating over the solid angles $\hat k_1, \hat k_2,\hat k~ \mbox{and} ~\hat K$ to finally obtain
\beq
\braket{k l j K L \mathcal J, T_z J}{k_a l_a j_a k_b l_b j_b T_z K}=\frac{4\pi^2}{kKk_ak_b}\delta(\omega)\theta(1-x^2)A(x)
\eeq
with
\beq
\begin{split}
\omega = k^2+\frac{1}{4}K^2-\frac{1}{2}(k_a^2 + k_b^2)\\
x=(k_a^2-k^2-\frac{1}{4}K^2)/kK
\end{split}
\eeq
and 
\beq
A(x)=\frac{1}{2\lambda+1}\sum_\mu [Y^l(\hat k)\times Y^L(\hat K)]^{\lambda*}_\mu \times[Y^{l_a}(k_a)\times Y^{l_b}(k_b)]^\lambda_\mu
\eeq
The functions \sd Y \sd\, are the spherical harmonics and \sd x\sd\, is the cosine angle between \sd \bold k\sd\, and \sd \bold K \sd. 
From Eq.~\eqref{eq:brackettransform} we obtain the expression for the interactions in 
laboratory coordinates as
\be
\begin{split}
&\bra{k_al_aj_ak_bl_bj_b T_zJ}v\ket{k_cl_cj_ck_dl_dj_dT_zJ}=\\
&\sum_{lLj\mathcal Jl'}\int d^3k \int d^3K\int d^3k'\bra{kljKL\mathcal J,T_zJ}v\ket{k'l'jKL\mathcal J T_zJ}\\
&\times \begin{Bmatrix}l_a & l_b & \lambda\\\frac{1}{2} & \frac{1}{2} & S\\ j_a & j_b & J \end{Bmatrix}
 (-1)^{\lambda+\mathcal J-L-S}F\hat{\mathcal J} \hat \lambda^2\hat j_a \hat j_b \hat S\begin{Bmatrix}L & l & \lambda \\ S & J & \mathcal J\end{Bmatrix}\\
& \times\braket{k l j K L \mathcal J, T_z J}{k_a l_a j_a k_b l_b j_b T_z J}\\
&\times \begin{Bmatrix}l_c & l_d & \lambda'\\\frac{1}{2} & \frac{1}{2} & S'\\ j_c & j_d & J \end{Bmatrix}
 (-1)^{\lambda'+\mathcal J-L-S'}F\hat{\mathcal J} \hat \lambda'^2\hat j_c \hat j_d \hat S'\begin{Bmatrix}L & l' & \lambda' \\ S' & J & \mathcal J\end{Bmatrix}\\
& \times\braket{k_c l_c j_c k_d l_d j_d T_z J}{k' l' j K L \mathcal J, T_z J}.
\label{eq:vvtillab}
\end{split}
\ee


\section{Interactions again}

In the last section we showed how to derive the interactions from relative coordinates. We observed that it is convenient to do the calculations in relative coordinates because the interactions are diagonal in the center of mass coordinates and in relative angular momenta. We will now show how we find the orbital momentum dependency in the interactions. 
The interactions are on the form
\beq
		\bra{\bold p}v\ket{\bold k}\notag.		
\eeq
We insert the completeness relation 
\beq
\int d^3r \ket{\bold r}\bra{\bold r}=I
\eeq
on both bra and ket side
\be
\int d^3r\int d^3r'\braket{\bold p}{\bold r}\bra{\bold r} v \ket{\bold r'}\braket{\bold r'}{k}.
\label{eq:kompletthet}
\ee
For a local potential $v$, we write equation \eqref{eq:kompletthet} as
\begin{align}
	&	\bra{\bold p}v\ket{\bold k}=\int d^3r \braket{\bold p}{\bold r}\bra{\bold r}v\ket{\bold r}\braket{\bold r}{\bold k}\notag\\
	&=\frac{1}{(2\pi)^3}\int d^3re^{-i\bold p \bold r}v(\bold r)e^{i\bold{kr}},
	\label{eq:dyttetinnkom}
\end{align}
where we have inserted for the definition
\beq
\braket{\bold p}{\bold r}=\frac{1}{(2\pi)^{\frac{3}{2}}}e^{-i\bold{pr}}.
\eeq
In chapter \ref{ch:coupled} we showed how plane waves can be expanded in partial waves, we expand  both of the exponentials
in Eq. \eqref{eq:dyttetinnkom},
\begin{align}
		&\frac{1}{(2\pi)^3}\int d^3r \sum_l (2 l + 1 ) i^l P_l(\Omega_{p,r}) j_l(pr) \sum_{l'} (2 l' + 1 ) i^{l'} P_{l'}(\Omega_{k,r}) j_{l'}(kr).
		\label{eq:soonpotential}
\end{align}
For a centrally symmetric potential, gives us the interaction on the form
\begin{align}
	&	\frac{1}{2\pi^2}\int r^2dr\sum_l (2 l +  1 ) P_l(\Omega_{p,k}) j_l(pr)vj_l(kr)\notag \\
		&=\frac{1}{2\pi^2}\sum_l (2 l +  1 ) P_l(\Omega_{p,k})\bra{pl}v\ket{kl}, 
		\label{eq:vrl}
\end{align}
where we have used the orthogonality properties of the Legendre polynomials, see appendix \ref{app:leg} for details.
In the presence of a tensor force it is the angular momentum, $j$ that is conserved and not the orbital momentum 
$l$. The interactions will be expressed as
\begin{align}
		&\frac{1}{2\pi^2}\sum_{jll'} (2 j +  1 ) P_j(\Omega_{p,k})\bra{pjl}v\ket{kjl'}.
			\label{eq:vrlj}
\end{align}
For each $j$ the orbital momentum in relative coordinates may have the values $|j-1|,j,$ and $j+1$. 
In our interactions we also included the total isospin, $T_z$, as a good quantum number.
