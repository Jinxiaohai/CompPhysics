\chapter*{Notation}

\section*{Units}
We will in all chapters except the second, regarding quantum mechanics, work with units, where
\beq
\hbar=c=1
\eeq
In this system,
\begin{align}
		&\mbox{[length]}=\mbox{[time]}=\mbox{[energy]}^{-1}=\mbox{[mass]}^{-1}.\notag
\end{align}
The mass $m$ of a particle is therefore equal to its rest energy ($mc^2$).

\section*{Vectors}
Vectors are boldfaced, $\bold v$ is an ordinary vector in three dimensions. Four vectors are denoted by a Greek letter, $A^\mu$, where $\mu=(0,1,2,3)$
\beq
A^\mu=\left(A^0,\bold A\right)~\mbox{and}~ A_\mu=(-A^0,\bold A)
\eeq


\section*{Summation convention}
We use Einstein summation conventions, where equal upper and lower indices are summed over, Greek letters are summed from zero to three,
\beq
A^\mu B_\mu=\sum_{\mu=0}^3A^\mu B_\mu.
\eeq

\section*{Spin and isospin}
The Pauli spin matrices $\mathbf\sigma$ are defined as
\beq
\sigma^i=\begin{pmatrix} 0 & 1\\1 & 0\end{pmatrix},~
		\sigma^j=\begin{pmatrix} 0 & -i\\i & 0\end{pmatrix}~\mbox{and}~
				\sigma^k=\begin{pmatrix} 1&0\\0&-1\end{pmatrix}.		
\eeq
The isospin matrices $\mathbf\tau$ are defined as 
\beq
\tau^i=\begin{pmatrix} 0 & 1\\1 & 0\end{pmatrix},~
		\tau^j=\begin{pmatrix} 0 & -i\\i & 0\end{pmatrix}~\mbox{and}~
				\tau^k=\begin{pmatrix} 1&0\\0&-1\end{pmatrix}.		
\eeq
\section*{Commutators}
A commutator between two operators $A$ and $B$  is defined as
\beq
[A,B]=AB-BA.
\eeq
The anticommutator is defined as
\beq
\{A,B\}=AB+BA.
\eeq
