\chapter{Introduction}
%%%%%%%%%%%%%%%%%%%%%%%%%%%%%%%%%%%%%%%%%%%%%%%%%%%%%%%%%%%%%%%%%%%%%%%%%%%%%%
In 1967 the first
pulsar was observed \cite{nat:hewish}, and based on characteristic
observational features this object was identified as a neutron star. After
direct evidences of the existence of neutron stars, nuclear models have been widely
employed in the description of the internal structur e of neutron stars. It 
turned out that the equation of state of nuclear matter is not only a very 
important ingredient in the study of nuclear properties and heavy ion 
collisions, but also in studies of neutron stars and supernovae.\\
\\
%%%%%%%%%%%%%%%%%%%%%%%%%%%%%%%%%%%%%%%%%%%%%%%%%%%%%%%%%%%%%%%%%%%%%%%%%%%%%%
Nuclear matter is an idealized system with an infinite amount of nucleons
and contains an equal amount of protons and neutrons. Even though it is a
theoretical construct it is possible to obtain some ``experimental'' values
regarding nuclear matter, such as the binding energy per nucleon and the saturation density, $\rho_0$, which
is a function of the Fermi momentum $k_f$. 
This is obtained by using the semi-empirical mass formula and divide
by the nucleon number, $A$, letting
$A$ go to infinity. A main purpose of nuclear matter theories is to derive the 
binding energy per nucleon by first principles. Following this approach one can determine the nuclear matter density $\rho_0$ and the incompressibility coefficient $K$ which relates to the equation of state through
\beq
K=\Big[k^2\frac{d^2}{dk^2}\big(\frac{\varepsilon}{\rho}\big)\Big]=9\Big[\rho^2\frac{d^2}{d\rho^2}\big(\frac{\varepsilon}{\rho}\big)\Big],
\eeq
where $\varepsilon$ denotes the energy density, $\rho$ is the Baryonic density 
and $k$ denotes the momenta. The incompressibility coefficient defines the 
curvature of the equation of state $\varepsilon(\rho)/\rho$ at $\rho_0$.\\ 
\\
%%%%%%%%%%%%%%%%%%%%%%%%%%%%%%%%%%%%%%%%%%%%%%%%%%%%%%%%%%%%%%%%%%%%%%%%%%%%%%%
%%%%%%%%%%%%%%%%%%%%%%%%%%%%%%%%%%%%%%%%%%%%%%%%%%%%%%%%%%%%%%%%%%%%%%%%%%%%%%
There are many theoretical reasons that motivate the use of coupled cluster. The
method is fully microscopic. When one expands the cluster operator in coupled cluster theory to all particles in a system, one reproduces the full correlated many-body wavefunction of the system.  The coupled cluster 
method is size consistent, the energy of two noninteracting fragments computed separately is the same as computing the energy for both systems simultaneously. Furthermore, the coupled cluster method is  size extensive, the energy computed scales linearly with the number of particles. A size extensive method is often defined as a method where there are no unlinked diagrams in the energy and amplitude equations.
The coupled cluster method is not variational, however the energy tends to behave as a variational quantity in most instances.\\
\\
%%%%%%%%%%%%%%%%%%%%%%%%%%%%%%%%%%%%%%%%%%%%%%%%%%%%%%%%%%%%%%%%%%%%%%%%%%%%%
%%%%%%%%%%%%%%%%%%%%%%%%%%%%%%%%%%%%%%%%%%%%%%%%%%%%%%%%%%%%%%%%%%%%%%%%%%%%%
The aim of the thesis is to do nuclear matter calculations with the coupled
cluster method.
We calculate the binding energy for nuclear matter.\\
%and try to obtain an equation of state for pure neutron matter.\\
\\
% The concept of nuclear matter came as a consequence of almost constant binding energies of nuclei, together with the observations of almost constant central densities of nuclei with nucleon number $A\geq 30.$ \\% With the
%semiempirical mass formula one was able to get an ``experimental'' value on the
%binding energy per nucleon for symmetric nuclear matter and one found that it
%to be approximately -16 MeV.
This is not the first work on nuclear matter, different many-body methods such
as Hartree-Fock calculations and perturbation theory have also been performed on
nuclear matter. However a perturbative approach is difficult because of the
repulsive core in the nucleon-nucleon interaction. This difficulty has been
circumvented by using Br\"ueckner's method, by defining the so-called
Br\"ueckner G-matrix.  Even the coupled cluster method has been used to
calculate properties of nuclear matter as done in Ref.~\cite{dayzab}.\\
\\
In our calculations the interaction-elements are given
in laboratory coordinates and the wavefunction expanded in partial waves.%  Calculations with laboratory
The calculations were done in a plane wave basis in the laboratory system, by
using transformation brackets described by Kung in Ref.~\cite{kung}.  When operating in a plane
wave basis it is necessary integrate over the momenta. The numerical integration was done by using twelve mesh points, six mesh points for holes and six for particles.
%coordinates is convenient, since we do not need to consider angles in the
%integrations and since the coupled cluster equations are defined in laboratory coordinates.\\

%In
%order to get interactions in laboratory coordinates we performed a bracket
%transformation from relative coordinates.
%The equation of state
%describes how the energy density and pressure vary with density and
%temperature.
As we wanted a more
theoretical approach to the problem we chose to use the interactions derived
from the chiral symmetries of QCD, N$^3$LO with the scale $\Lambda=500$ MeV, 
rather than using the more
phenomenological ones. The N$^3$LO potential was further renormalized with a similarity transformation method, resulting in a so-called low-momentum interaction V$_{low-k}$. We wanted to calculate with at least three cutoffs, $\lambda=2.1$, $\lambda=2.2$ fm$^{-1}$ and $\lambda=2.5$ fm$^{-1}$.\\
% We wanted to do calculations with $\lambda=3.0$ fm$^{-1}$ as well, however because of time limits it become impossible. \\
\\
We managed to calculate energies for the cutoffs at $\lambda=2.1$ fm$^{-1}$ and $\lambda=2.2$ fm$^{-1}$. With the cutoff $\lambda=2.5$ fm$^{-1}$ we were just able to compute the energy for one $k_f$ value, because of both time limits and convergence problems.
%You have to write about the potentials used in the programs, the potentials
%that I am supposed to use in the calculations.
\section*{Outline}

Large parts  of the thesis contain a description of the theoretical
prerequisites. The first chapter gives a brief review of quantum mechanics, 
which is thought to be a natural theory to include. Instead of giving a mathematical
definition of quantum mechanics we preferred to write more about the
philosophical interpretations of it, such that not only physicists will enjoy
reading it. The following chapter gives a short overview of second
quantization which culminates in normal ordering the Hamilton operator, a feature  which is
crucial in the coupled cluster calculations.  Since we are doing  a many-body
calculation we found it rather important to include a chapter on perturbation
theory. Actually, some of the diagrams obtained in the perturbative approach are
similar to the ones in the coupled cluster approach. Since we are doing calculations
of nuclear matter we felt it impossible not to write about the nuclear force,
and we go quickly through chiral perturbation theory since we are 
using interactions derived from it. Chiral perturbation theory is a rather hard 
subject and the author admits that he is still not acquainted with it. Of course there is also a chapter only dedicated to
the coupled cluster method. We have tried to write most of the derivations, 
since it is not uncommon to become rather frustrated when writers
leave out crucial derivations in their books. We refer to other texts  whenever parts of the derivations are left out, especially 
Ref.~\cite{sjefer} for which chapter \ref{ch:coupled} is based on.  The final part concerns the results of  the calculations.\\
\\
%at first I wanted to truncate it at the value 6, in the
%laboratory system, however some unforeseen problems arose, which was that the
%computer cluster Titan at the university of Oslo was upgraded in the middle of
%the semester which resulted in a stop and further problems with the file system
%so I decreased the truncation to four, just to be able to finish in time. However this
%affects the result and I can not say it enough that with a maximum value of 6, I
%would probably have increased the precision of the binding energy, although the
%qualitative shape of the curve seems good. 
