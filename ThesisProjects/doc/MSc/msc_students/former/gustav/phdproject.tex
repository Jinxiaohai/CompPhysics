\documentstyle[a4wide]{article}
\newcommand{\OP}[1]{{\bf\widehat{#1}}}

\newcommand{\be}{\begin{equation}}

\newcommand{\ee}{\end{equation}}
\newcommand{\bra}[1]{\left\langle #1 \right|}
\newcommand{\ket}[1]{\left| #1 \right\rangle}
\newcommand{\braket}[2]{\left\langle #1 \right| #2 \right\rangle}


\begin{document}

\pagestyle{plain}

\section*{Possible PhD thesis project for Gustav Ragnar Jansen}

The thesis project embraces many topics, from the basic understanding of the strong force in nuclei to
various mathematical algorithms needed to solve Schr\"odinger's equation with many interacting particles.
As such it fits excellently under the 'computational quantum mechanics' activity of CMA.
The mathematical methods will focus on non-linear systems of coupled equations, Monte Carlo methods applied to Lattice quantum-chromo-dynamics calculations and partial differential equations.  In addition, the effective usage of scripting languages
in computational science will also be an issue, in particular how to efficiently parallelize codes written in python 
that interface Fortran95 and C++ codes. 

The main focus is on the  physics of hypernuclei, with the Coupled-cluster method as the many-particle method of
choice for solving Schr\"odinger's equation with many interacting particles.

Hypernuclei are bound systems of neutrons, protons and one or more strange
baryons, such as the $\Lambda$ or $\Sigma$ hyperons. Understanding the
behavior of hypernuclei (how they are produced, their spectroscopy and
decay mechanisms) has been the subject of intense investigations
during the last decades.  There are large experimental upgrades worldwide, with nex experimental
facilities available from 2010 and on in Japan, Europe and the U.S.A.

One of the main goals of such studies is to explore how the presence
of the new degree of freedom (strangeness) alters and broadens the
knowledge achieved from
conventional nuclear physics. Furthermore, since nuclei with hyperons involve particles with strange quarks
in addition to the up and down quarks making up neutrons and protons, studies of hypernuclei offer experimental and 
theoretical insights on the strong force originating from quantum chromodynamics.
It is a field which holds great promise for large
discoveries in the future, with several new facilities planned worldwide.


In studies of hypernuclei, many-particle calculations are still in  a
very preliminar stage, in particular when it comes to studies
of weakly bound systems, of current experimental interest both at FAIR at GSI in Darmstadt
and J-PARC at  KEK  in Japan. To provide reliable ab initio calculations for weakly bound
hypernuclei is a great challenge and this thesis is placed in such a 
context.  Since the systems are normally weakly bound, it means that one needs to account for 
states which can be resonances, that is states which have a finite width (imaginary energy).


The starting point for these studies  is the derivation of 
 new effective interactions for hyperons interacting with 
nucleons in many single-particle orbits. This defines the space for 
a self-consistent Hartree-Fock calculations of hyperons and nucleons.
The numerical methods used are based on diagonalization techniques and
similarity transformations.  These calculations, which were part of Gustav Jansen's Master thesis, will then be used as input
to large-scale many-particle studies of hypernuclei.

Based on this, this thesis project aims at 
extending the developed code to treat systems where
some of the states are weakly bound or in the scattering continuum.

The next step is to use the interactions developed in the previous 
step and perform many-body calculations with our recently developed
Coupled-Cluster codes.  It will be the first ab initio calculation of weakly
bound hypernuclei and could provide benchmark calculations for new and planned
experiments.  
The role of three-body forces for hypernuclei will also be studied.
Finally, since the interaction between nucleons and hyperons is not well-known, recent advances in Lattice
quantum chromodynamics (QCD) calcualtions allow now for the possibility to extract a functional form for 
such interactions. These interactions can then be used as the basic Hamiltonians for large-scale coupled-cluster
calculations.  This part will be done in collaboration with Tetsuo Hatsuda's group at the University of Tokyo.
This group has recently performed Lattice QCD calculations  from which they can extract the shape of the nucleon-nucleon
interaction at a given energy.

The numerical methods for the coupled-cluster codes are 
based on the iterative solution of non-linear equations
which are parallelized. 
These codes can also be extended to ab initio calculations of other systems of fermions, such as 
quantum dots.
Another aim of this thesis is therefore to develop general  many-particle codes
for ab initio studies of different fermionic systems.

The activity will be based within the 
Computational Quantum Mechanics project at the Department of Physics and 
CMA. 
The Computational Quantum Mechanics group consists  
presently of two full professors (Hjorth-Jensen and Osnes), one professor Emeritus (Engeland), 
one Adjunct Professor (Dean, Oak Ridge National Lab and CMA), four PhD
students (CMA(2), Dept ot Physics (2)) and ten Master students (Dept of Physics).  

The group has strong collaboration with Oak Ridge National Laboratory, Michigan State University and 
the University of Tokyo. It is expected that Gustav will spend parts of his PhD at Oak Ridge National
Laboratory and the University of Tokyo with Prof. Tetsuo Hatsuda's group on Lattice quantum chromodynamics.


\end{document}


Metoder - Similaritets trans, Coupled cluster, Lattice QCD, Effektive vv.
Quantum many-particle problem with a more general set of particles.
Fokus p� generelle metoder, som kan brukes p� tvers av fagfelt:
Coupled cluster - mot DFT, kjernefysikk og mulige andre?
Monte Carlo - matematisk finans, PDE, Lattice QCD...
Uendelig kjernematerie med hyperoner -> Neutronstjerner
Bruk av Lambda for � f� kjerner med flere n�ytroner/protoner
Mange nye eksperiment fremover
Tre-partikkel krefter ink. hyperoner
Link mot materialfysikk, nanotechology og quantum dots (Coupled cluster/DFT?)

Vil gj�re:
Coupled cluster for hyperkjerner, ink 3-partikkel krefter
Bruke fortran til intensive deler, men lage bibliotek til Python for admin,
visualisering mm. Muligheter for web-tjeneste ala nn-online, med beregning av
eff. vv.
Fokus p� generell kode, slik at det om mulig kan brukes mot kvantekjemi
Fokus p� parallellisering mot forskjellige hw plattformer

Lattice QCD for hyperoner for � lage bedre 2-part. vv.
Viktig pga. snart nok regnekraft til � l�se kvantitativt.
Monte carlo metoder for l�sing av funksjonelle integraler
Mulig mer generellt fokus med bruk av MC metoder innenfor andre omr�der.





























